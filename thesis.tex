%% FEUP THESIS STYLE for LaTeX2e
%% how to use feupteses (English version)
%%
%% FEUP, JCL & JCF, 31 July 2012
%%
%% Read the documentation inline 
%% and at https://web.fe.up.pt/~jlopes/doku.php/doc/teach/feupteses
%%
%% PLEASE send improvements to jlopes at fe.up.pt and to jcf at fe.up.pt
%%

%%========================================
%% Commands: pdflatex tese
%%           bibtex tese
%%           makeindex tese (only if creating an index)
%%           pdflatex tese
%% Alternative:
%%          latexmk -pdf tese.tex
%%========================================


%% 2021-07-20: One-sided output by default
\documentclass[11pt,a4paper]{report}
%% For two-sided printing (for dead-tree output) comment previous line
%% and uncomment the next line
%% \documentclass[11pt,a4paper,twoside,openright]{report}

%% MEIC options
% \usepackage[meic]{feupteses}
%\usepackage[meic,juri]{feupteses}
%\usepackage[meic,final]{feupteses}
%\usepackage[meic,final,onpaper]{feupteses}
\usepackage[meic,juri,backrefs]{feupteses}

%% MEEC options
%\usepackage[meec]{feupteses}
%\usepackage[meec,juri]{feupteses}
%\usepackage[meec,final]{feupteses}

%% For other degrees
%\usepackage{feupteses} % you must define the degree bellow

%% Additional options for feupteses.sty: 
%% - onpaper: links are not shown (for paper versions)
%% - backrefs: include back references from bibliography to citation place

%% Uncomment to create an index (at the end of the document)
%\makeindex

\usepackage[utf8]{inputenc}

\usepackage{xspace}
\xspaceaddexceptions{\%}
\xspaceremoveexception{-}

% *** GRAPHICS RELATED PACKAGES ***

\graphicspath{{figures/}}

\usepackage{tabularx,ragged2e}
\usepackage{etoolbox}

\usepackage{tikz}

% *** HYPHENATION ***

\usepackage{soul} % Hyphenation for letterspacing, underlining, and more

% *** SPECIALIZED LIST PACKAGES ***

\usepackage{paralist}
\usepackage{verbatim}
\usepackage{listings}

\lstset{
  basicstyle=\small\ttfamily,
  columns=flexible,
  breaklines=true
}

\usepackage{enumitem}
\setitemize{noitemsep,topsep=1pt,parsep=1pt,partopsep=1pt}

% *** PDF, URL AND HYPERLINK PACKAGES ***

% \usepackage[hyphens]{url}
\usepackage{hyperref}
\hypersetup{
  pdftitle={},
  pdfauthor={},
  colorlinks=true,
  bookmarks=false,
  bookmarksopen,
  bookmarksnumbered
}
\def\UrlBreaks{\do\/\do-}

\usepackage[hang,flushmargin]{footmisc}

% *** CITE AND REF PACKAGES ***

\usepackage[capitalise,noabbrev]{cleveref} % Must be loaded *after* hyperref
\newcommand{\creflastconjunction}{, and\nobreakspace} % use serial comma
\usepackage[numbers,sort]{natbib}
% \usepackage{balance}

%%----------------------------------------
%% TIP: if you want to define more macros, use an external file to keep them
\include{mymacros}

%% Custom commands

\newcommand{\etal}{et al.\xspace}
\newcommand{\todo}[1]{{\color{red}\bfseries [[#1]]}}

%%----------------------------------------

%%========================================
%% Start of document
%%========================================
\begin{document}

%%----------------------------------------
%% Information about the work
%%----------------------------------------
\title{Automatic Detection of Semantic Conflicts in Merge Commits via LLMs}
\author{Duarte Guedes Sardão}

%% Comment next line if not necessary for degree
%\degree{Programa Doutoral em Engenharia Informática}

%% Uncomment next line for date of submission
%\thesisdate{July 31, 2008}

%% Comment next line copyright text if not used
%\copyrightnotice{Name of the Author, 2008}

\supervisor{Supervisor}{José Campos}

%% Uncomment next line if necessary
%\supervisor{Second Supervisor}{Name of the Supervisor}

%% Uncomment committee stuff in the final version if used
%\committeetext{Approved by \ldots:}
%\committeemember{President}{Name of the President}
%\committeemember{Referee}{Name of the Referee}
%\committeemember{Referee}{Name of the Referee}

%% Uncomment signature line in the final on paper version if used
%\signature

%% Specify cover logo (in folder ``figures'')
\logo{uporto-feup.pdf}

%% Uncomment next line for additional text below the author's name (front page)
%\additionalfronttext{Preparação da Dissertação}

%%----------------------------------------
%% Preliminary materials
%%----------------------------------------

% remove unnecessary \include{} commands
\begin{Prolog}
  \chapter*{Resumo}
%\addcontentsline{toc}{chapter}{Resumo}

Em desenvolvimento de software colaborativo, trabalho paralelo, desenvolvido em diferentes ramos, tem de ser frequentemente integrado. Devido às diferenças do trabalho efetuado nos diferentes ramos, conflitos surgem frequentemente. Alguns conflitos são simples de detetar e rectificar, como conflitos de integração textuais, que surgem quando diferentes desenvolvedores editam a mesma linha em ramos diferentes: a maior parte de sistemas de controlo de versões consegue automaticamente detetar que a mesma linha foi alterada e impele o integrador a decidir numa solução: que alteração manter e que discartar.

Entre conflitos de integração, os semânticos emergem com um tipo particularmente difícil de resolver, porque não são detetados por sistemas de controlo de versões e, não tendo erros sintáticos, compilam com sucesso. 
Um exemplo de um conflito semântico pode ser visto na integração de mudanças paralelas de uma classe Point, que contêm um método para calcular a distância para outro Point: num cenário em que um desenvolvedor num ramo A muda o cálculo de distância (de euclideana para manhattan, por exemplo), enquanto que no B outro introduz uma função move(), que usa o cálculo da distância como o valor para o movimento. Percebemos que depois de a integração nenhum erro é levantado, mas o comportamento está desviado dos ramos originais: especificamente, o movimento comportar-se-á de maneira diferente.

Exploramos como o emergente ramo de Grandes Modelos de Linguagem pode fornecer um enorme avanço na nossa abilidade de testar o comportamento alterado, introduzido ou perdido devido a conflitos semânticos.
Especificamente, analisamos a abilidade de ChatGPT para descrever o conflito presente numa integração, bem como a capacidade de gerar testes unitários apropriados that evidenciam a presença de conflitos semânticos, com a descrição fornecida.
Concluímos que, geralmente, o ChatGPT demonstra capacidade para ambas as tarefas quando apresentado código de pouca complexidade. Outro obstáculo prominente é a associação entre ``conflito'' e falha de software ativa, que leva a 
dificuldade a identificar conflitos mais simples.

\bigskip\noindent
\textbf{Palavras-chave:} Grandes Modelos de Linguagem (LLMs)\todo{traduzir para PT}, ChatGPT, Conflitos Semânticos, Geração de Testes Unitários

% ------------------------------------------------------------------------------

\chapter*{Abstract}
%\addcontentsline{toc}{chapter}{Abstract}

In collaborative software development, parallel work done by several different branches often has to be merged. Due to the differences of work done in the difference branches, often conflicts arise. Some conflicts are easy to detect and rectify, such as textual merge conflicts, where different developers have altered the same line in different branches: most version control systems can detect that the same line has been changed and urge the merger to decide on a solution, for example, by keeping one change and discarding the other.

Among merge conflicts, semantic merges arise as particularly difficult to resolve, as they avoid detection by version control systems and lacking syntactic errors, compile successfully.
An example of a semantic conflict can be seen in the integration of parallel changes in a class such as Point, containing a method to calculate the distance to another Point: one developer in branch A changes the distance calculation (from euclidean to manhattan, for example), while in B another introduces a move() method, which uses the distance calculation as the value to move by. We find that upon a merge, while no errors are raised, the code exhibits altered behaviour from the original branches: specifically, the movement will behave differently.

We sought to explore how the emerging field of Large Language Models can provide a breakthrough in our ability to test for the altered, introduced and lost behaviours arising from semantic conflicts. 
Specifically, we analysed the ability of ChatGPT to describe the conflict present in a merge, as well as it capabality to generate appropriate unit tests that highlight the presence of semantic conflicts, with the description provided.
We concluded that, in most cases, ChatGPT demonstrates capacity for both tasks when presented with low complexity code. Another notable obstacle is the association between ``conflict'' and active software fault, which leads to difficulty identifying simpler conflicts.

\bigskip\noindent
\textbf{Keywords:} Large-language Models (LLMs), ChatGPT, Semantic Conflicts, Unit Test Generation
 % the abstract
  \chapter*{Acknowledgements}
%\addcontentsline{toc}{chapter}{Acknowledgements}

I would like to express my sincere gratitude to my thesis coordinators, José Campos and Antónia Lopes, for their continuous support and invaluable guidance.
Their insights and advice in our regular meetings were instrumental, providing me with clear goals and targets, and I am deeply appreciative of their contributions.

I would also like to deeply thank my family, whose support and encouragement has been a constant source of strength throughoutmy entire academic journey. 
It was because of their love, education and support that I reached this milestone.

A special acknowledgment goes to Aalto University and the country of Finland, where I had the privilege to spend my first semester of this academic year.
My experience there, concurrently with the beginning of my work on this thesis, broadened my academic and personal experience.

Lastly, I would like to thank my friends who were there throughout this process.
Particularly I'd like to thank Ignasi, who accompanied me in Finland and in a vacation in March,
which gave me motivation to keep working and renewed my energies. 


\vspace{10mm}
\flushleft{Duarte Guedes Sardão}
  % the acknowledgments
  \cleardoublepage
\thispagestyle{plain}

\vspace*{8cm}

\begin{flushright}
   \textsl{``It takes a long time sometimes\\
   It can take a terrible long time before things sort themselves out.''} \\
\vspace*{1.5cm}
           Tove Jansson
\end{flushright}
    % initial quotation if desired
  \cleardoublepage
  \pdfbookmark[0]{Table of Contents}{contents}
  \tableofcontents
  \cleardoublepage
  \pdfbookmark[0]{List of Figures}{figures}
  \listoffigures
  \cleardoublepage
  \pdfbookmark[0]{List of Tables}{tables}
  \listoftables
  \chapter*{Abbreviations and Symbols}
%\addcontentsline{toc}{chapter}{Abbreviations}
\chaptermark{ABBREVIATIONS AND SYMBOLS}

\begin{flushleft}
\begin{tabular}{l p{0.8\linewidth}}
LLM      & Large Language Model\\
VCS     & Version Control System\\
DSL     & Domain Specific Language\\
\todo{are there others to be included here?} & \\
\end{tabular}
\end{flushleft}

  % the list of abbreviations used
\end{Prolog}

%%----------------------------------------
%% Body
%%----------------------------------------
\StartBody

%% TIP: use a separate file for each chapter
\chapter{Introduction} \label{chap:intro}

The purpose of this chapter is to introduce the motivation for the work, briefly describe the problem at hand and outline the work that will be developed, as well as the structure of the thesis.

\section{Motivation} \label{sec:motivation}

Collective software development requires the handling of merge conflicts, as conflicts between parallel work arise. Called merge conflicts, as they arise when this parallel work is merged, they vary in their difficulty of detection.
Common textual conflicts, where the same line is altered by multiple people, are automatically detected by version control systems, allowing amendments to be easily made. However not all conflicts are so easily detected and their occurence can bring with it the addition of software bugs to the system. Semantic merge conflicts, in particular, are hard to detect, both by software and human review and remain a hard to solve situation. Given that around 30\% of developers do not even actively monitor for merge conflicts and of those who do, mostly do it with reactive strategies~\cite{kn:lifecycle}, the ability to automatically detect these would be a great boon for the field of software development, especially as by developers own admissions, the longer a merge conflict is left unsolved, the harder it becomes to resolve it: ``Untangling takes days instead of minutes when it gets too out of hand.''~\cite{kn:lifecycle}.

The recent revolution in the field of Large Language Models (LLMs) may prove to add a valuable tool in tackling this, given their ability to describe the functioning of code snippets as well as generate them. These capabalities may allow us, with the correct development and prompting, to not only describe the semantic conflict present in a specific merge commit, but also to generate the appropriate unit test that identifies the lost or emergent behaviour associated with it.


\section{Problem} \label{sec:problem}

To manage the concurrent work of several developers in software projects, it is common to employ \emph{version control systems} (hereby referred as \emph{VCS}), which can be defined as ``a system that manages the development of an evolving object. In other words, it is a system that records any changes made by the software developers.''~\cite{kn:vers_review}.

A significant task of version control is managing access to shared resources.
With ``pessimistic locking'' a lock-modify-unlock paradigm was adopted, where a given file would be locked for modification while it is being modified, thus ensuring each resource can only be handled by one actor at a time. VCS's however, generally implement a copy-modify-merge mechanism: concurrent work can done on a resource, with joining the parallel work together handled by merges afterwards, with two ``branches'' of work merged into one~\cite{kn:vers_ott}.

Merge conflicts arise when parallel work cannot be automatically merged. Of this, several different types exist, as summarized by \citet{kn:tmens}:

Textual conflicts occur when the same textual elements of code are modified in both branches of a merge. For example, when the same line of code is modified by two people in their respective branches.

Syntactic conflicts arise from parallel changes that when merged do not generate textual conflicts, but the resulting merge creates code that is invalid given the languages rules. For example, programmer A renames a variable, while programmer B uses the variable somewhere (with the original name). There is no textual conflict, but the code will not compile due to the usage of an uninitialized variable.

Finally semantic conflicts occur when parallel changes do not have any textual conflict and their merge is syntactically valid, but the resulting code does not behave as expected, or exhibits lost or new unexpected behaviour.

Most VCS's, such as Git, implement textual merge tools (ergo, they can only identify textual conflicts). However there are specialized tools that handle other types of merges. for instance Turbomixer is also able to handle syntactic merges~\cite{kn:tmens}. The focus on textual merging is unsurprising as around 90\% of conflicts are textual~\cite{kn:lcsd} and syntactic conflicts are easy to identify after merging, as errors will be clearly indicated and programs will not compile.
Semantic conflicts remain as both undetected by VCS's and hard to detect after merges. Thus, finding methods to automatically identify and highlight semantic conflicts in merge commits has been a persistent problem and a source of study in the field~\cite{kn:nuno,kn:leuson,kn:leuson2}.

As a simple example of a merge conflict, consider a class ``Cart'' of a shopping app, subject to concurrent changes in two branches, as shown in \Cref{fig:conflict}.
Initially Cart just has a method total\_cost that calculates the total cost given a percentage of discount.
In branch A the existing method is modified to take both tax and discount as parameters and the method is overloaded by a new version that only takes the tax parameter.
In branch B, a checkout method is added, which calls the original total\_cost method. After merging, no error appears, but the checkout function will be running the total\_cost tax method, expecting the calculation of a discount. The example is shown in \Cref{fig:conflict}.

From here the conflict arises, as after merging the behaviour of the checkout function is not the intended one: instead of calculating the cost with a discount, it now uses the value given for discount as a tax.
To identify this conflict, we would need to generate a test that would find evidence of this emergent behaviour: in this case with a test for checkout that calculates the expected cost with tax. This test would pass on the merge commit and fail in Branch B and not compile in A (as checkout is not introduced),
certifying that due to the semantic conflict, new unexpected behaviour has been introduced.

\begin{figure}[t]
    \includegraphics[width=0.86\textwidth]{conflict}
    \caption{Model of the motivating example conflict, with Base, Branch A, Branch B and Merge versions of the code.}
    \label{fig:conflict}
\end{figure}

\section{Goal} \label{sec:goal}

The goal of this research project is to assess to what extent existing Large Language Models can be used to identify semantic conflicts, both textually and through the generation
of appropriate unit tests. By comparing our results to existing prompts for unit test generation, we will assess to what extent, if any, we improved on and how to further develop
unit test generation for semantic conflicts using LLMs.
\todo{review}
\section{Study} \label{sec:study}

Following an initial exploratory phase, where different prompts, techniques and models were experimented with, our work approaches the issue by answering
3 research questions:

1. To what extent can a LLM textually explain a semantic conflict present in a merge.

2. To what extent can a LLM generate appropriate unit tests, given a textual explanation of the semantic conflict. 

3. To what extent do existing prompts generate appropriate unit tests and how do they compare to the prompt we developed. 
\todo{review}

\section{Thesis Structure} \label{sec:struct}


\Cref{chap:intro} introduces the topic and problem at hand.
\Cref{chap:rw} details existing related work on the topics at hand.
\Cref{chap:study} described our study and research questions and \Cref{chap:results} answers each research question.
\Cref{chap:conclusion} gives a conclusion.

\chapter{Related Work} \label{chap:rw}

The following sections introduce the two main topics dealt with in this dissertation (merge conflicts and the usage of large language models for software verification and test generation). They explore related work and how we can build on it to develop our approach.

\section{Semantic Conflict Detection}

The study of merge techniques and conflicts has a long history, likely even predating the specific terminology itself. Thus there is a large corpus of work to explore.

Several related works exist which seeks to develop methods that can more systemically identify adulterated behaviour arising from semantic conflicts.

\subsection{Detection without Testing}

Several solutions exist which attempt to identify the presence of semantic conflicts without generating unit tests. An example of this rests on verifying overriding assignment: if a merge was successful where the same variable was assigned in both A and B, it's likely this is unintentional and there is a conflict. Thus, the solution checks if such a situation happens and reports if so to the developers~\cite{kn:assigne}.


Static analysis solutions have been considered, implementing the previous overriding assignment checks, inter procedural data flow and program dependence graph, which try to find data and control flows between the changes done by both developers, as well as confluence, which checks if there is data and control flows from both branches that flow into some common path. This static analysis shows much better F1 score and recall than dynamic analysis techniques, but much worse precision~\cite{kn:staticanal}.


The tool DeltaImpactFinder compares the impact of a change in the origin and destination branches, calculating the difference, the ``Delta-Impact''. To measure impact, dependencies are mapped out: the assumption is that when an entity is changed its dependencies are impacted. A semantic conflict is identified if there are missing or extra dependencies and this can be identified with the calculation of the delta-impact~\cite{kn:deltaimpact}.

The notion of semantic conflict-freedom is seen as a sufficient condition for correctness. The SafeMerge~\cite{kn:safemerge} tool seeks to verify the presence of semantic conflict-freedom in the merge process, avoiding the creation of semantic conflicts. Semantic conflict-freedom can be defined, for branches O (Origin), A, B and M (Merge), for all i inputs:
\begin{itemize}
  \item If out(O,i) $\neq$ out(A,i), then out(M,i) = out(A,i)
  \item If out(O,i) $\neq$ out(B,i), then out(M,i) = out(B,i)
  \item Otherwise out(O,i) = out(A,i) = out(B,i) = out(M,i)
\end{itemize}

\subsection{Detection with Testing}

In identifying the presence of semantic conflicts in a merge conflict, developed solutions have focused on the automatic generation of unit tests.

By Da Silva et al., we find an attempt at identifying cases of semantic conflict by applying automated behaviour change detection~\cite{kn:leuson}. In summary, with a base commit B, a left L, right R and merge M, they observe that a generated unit test that passes in L but fails in B partially reveals the effect of the changes made in that branch. If the test then fails in M, it is likely the changes made in R interfere. To generate unit tests they used EvoSuite~\cite{kn:evosuite} and Randoop~\cite{kn:randoop}.

In their analysis they found that the developed tool only detected interference in four out of 15 changes within merge scenarios that do actually suffer from interference, corresponding then to a recall of 0.267. While this is a very modest rate, it displayed no false positives (precision of 1) and thus could likely be integrated in a testing process to prune possible merge conflicts early, or further studied and refined.

 Building upon their previous work, \citet{kn:leuson2} proposed SAM (SemAntic Merge), a tool that generates tests upon merges in Java.\footnote{SAM can be found at \url{https://github.com/leusonmario/SAM}.}. In summary, SAM initially does a simple textual merge to integrate the difference branches while identifying possible textual conflicts. After merging four program versions are built, to fully describe the merge scenario under test: Base, Left, Right, Merge. Source code transformations to improve testability are also part of the process, but optional. Finally, the test generation tools are fed objects serialized during the execution of existing test suites. After applying four test generations tools: EvoSuite~\cite{kn:evosuite}, Differential EvoSuite, Randoop~\cite{kn:randoop} and Randoop Clean, their own adapted version of Randoop, SAM executes the generated tests against the four versions of the program, identifies which tests failed, interpreting it with pre-defined interference criteria heuristics and from there reports conflicts, if detected~\cite{kn:leuson2}.

Detecting nine out of 28 conflicts, it shows improvements over previous work: the authors specifically highlight the best performance when combining tests from only EvoSuite and Differential EvoSuite. Regarding behaviour changes specifically, 89 are found. 
In both cases, they highlight the ability of transformations (for example, making private fields public) to increase testability, showing moderate improvements in some tested scenarios, with 20 additional changes and three additional conflicts detected (Differential Evosuite detects 3, each other generator detect one conflict each).  Regarding the 19 false-negatives, 11 of them showed behaviour changes, which were not caused by the changes due to the semantic conflicts~\cite{kn:leuson2}.


\citet{kn:nuno} has proposed the tool UNSETTLE (aUtomatic uNit teSt gEneraTion for semanTic confLict dEtection)\footnote{UNSETTLE can be found at \url{https://github.com/conflito/unsettle}.} which is composed of two modules:
%
\begin{itemize}
  \item \textbf{Changes-Matcher} module that identifies the possible presence of semantic conflicts, by first computing the changes between different versions (base and variants) and then comparing it to a set of patterns (listed in \Cref{table:pattern-table}) describing common sources of conflicts as a base. From this it generates a DSL file, highlighting which methods and classes should be put under test to identify the conflict.

  \item \textbf{Test generator module}, a modified version of EvoSuite that takes the previously created artifact as an input to guide test generation.
\end{itemize}
%
Of particular interest to us is the Changes-Matcher module, as this is also the starting point for our work, with the usage of a LLM over EvoSuite for test generation instead.

\begin{table}[t]
\setlength\extrarowheight{2pt}
\begin{tabularx}{\textwidth}{llX}
 \toprule
 \textbf{Group} & \textbf{ID} & \textbf{Description} \\
 \midrule
 Change Method  & CM & Update two different dependencies of a method or update one method and concurrently update one of its
dependencies \\ 
 \hline
 Change Method
and Field & CMF & Change the type of one field to a type that does not
override a method while a dependency for the method
is added to a method that reads the field
 \\
 \hline
 Dependency
Based & DB & Update a method while a dependency to it is added
concurrently \\
 \hline
 Field Hiding & FH & Hide the field of a superclass in a subclass and concurrently add a method in the subclass that writes the
super field
 \\
 \hline
 Overload by Access Change & OAC & Change the visibility of an overloaded method and
concurrently add a dependency to it \\ 
 \hline
 Overload by Addition & OA & Overload a method and concurrently add a dependency to it \\  
 \hline
 Parallel Changes & P & Concurrent changes to the same entity, i.e., method
(PM), constructor, (PC) or field (PF) \\ 
 \hline
 Remove Overriding & RO & Remove the override of a method and concurrently
add a dependency to it \\  
  \hline
 Unexpected Overriding & UO & Override a method in a subclass while a dependency to it is added concurrently (AO) or override an
Object-inherited method and concurrently add a dependency to it (UO) \\
 \bottomrule
\end{tabularx}
\caption{\label{table:pattern-table}Semantic Conflict Patterns identified by Changes-Matcher~\cite{kn:nuno}.}
\end{table}


Ti Jao et al have proposed test oracles for program merges. Most significantly, not only do they support two and three-way merges but also octopus merges (merge with any number of branches). The developed tool, TOM (testing on merges), generates tests to identify unexpected and lost behaviour. For this, they implement diff-line as a criterion, guiding the tests to cover lines that have been modified between the different program versions, generating different assertions for different versions. Stability checking is done: the test is rerun five times before being handed to the developers. The tool identifies 45 three-way and 87 octopus merges, from a universe of 389 of each~\cite{kn:ji2022}.

\Cref{table:tool-comparison} shows a comparison between some key features of the presented solutions.


\begin{table}[t]
\setlength\extrarowheight{2pt}
\begin{tabularx}{\textwidth}{p{0.15\textwidth}p{0.17\textwidth}p{0.10\textwidth}p{0.13\textwidth}p{0.305\textwidth}}
  \toprule
  \textbf{Tool} & \textbf{Recall} & \textbf{Octopus Merge Support} & \textbf{Publicly Available} & \textbf{Test Generator} \\
  \midrule
  SAM & 43\% \newline (12 out of 28) & No & Yes & EvoSuite, Differential EvoSuite, Randoop, Randoop Clean \\ \hline
  UNSETTLE & 35\% \newline (6 out of 17) & No & Yes & EvoSuite \\ \hline
  TOM & 17\% \newline (132 out of 778) & Yes & No & EvoSuite \\
 \bottomrule
\end{tabularx}
\caption{\label{table:tool-comparison}Comparison of SAM~\cite{kn:leuson2}, UNSETTLE~\cite{kn:nuno}, and TOM tools~\cite{kn:ji2022}.}
\end{table}


% 12 out of 28 SAM
% 132 out of 778 TOM
% changes matcher always finds it, tests are 6 out of 17

\subsubsection{Conflict definition to assess tests}\label{chap:rw:behaviours}

To assess whether a conflict is correctly identified with unit tests, we need to select a definition that relates to testing.
Thus, two concepts are usually considered: emergent behaviour and lost behaviour. Respectively this refers to behaviour that 
is introduced in the process of merging, despite not being present in either branch, and behaviour that exists in one branch
but is then lost in the merge. While several existing works have established formal definitions for these behaviours~\cite{kn:taoji,kn:leuson},
we focus on the definition formulated by~\citet{kn:nuno}, as they broadened previous formulations to consider cases where the test does not apply,
ie, cases where the test does not compile. This is common when a conflict includes a method that was introduced in one branch: a test generated including
it will compile for that branch and merge, but not the other branch or base. A case of this is the example presented in the introduction \Cref{fig:conflict}: to test the emergent behaviour we would need to test checkout, and this test is not applicable for base and branch A.

Thus, the behaviours are defined as:

\textbf{Emergent Behaviour}: Let \texttt{t} be a successful test for the merged program \texttt{M} with \texttt{$V_{1}$}
and \texttt{$V_{2}$} as its parents. We then say that \texttt{M} has emergent behaviour if \texttt{t} applies to both
parents and fails over the same assertion \texttt{$\psi$} in both, or \texttt{t} applies to a single parent and fails.

\textbf{Lost Behaviour}: Suppose that the parent versions \texttt{$V_{1}$} and \texttt{$V_{2}$} have the common ancestor \texttt{B}.
Let \texttt{t} be a successful test case for \texttt{$V_{i}$}.
If \texttt{t} applies to \texttt{B} and \texttt{M} and \texttt{t} fails in both over the same assertion \texttt{$\psi$}, or \texttt{t} does not apply to \texttt{B} and fails in \texttt{M},
or \texttt{t} does not apply to \texttt{B} and \texttt{M}, then we say that the behaviour introduced by \texttt{$V_{i}$} was lost.


\section{Test Generation}

\subsection{Traditional Automatic Test Generation}

Traditional methods of automated test generation can be divided into random and search-based techniques. The former, being random, is simpler and faster, while the latter employs heuristics and search algorithms to fulfil some criteria, like maximizing code coverage.

Randoop is one such example of random test generation. Some of the disadvantages of this technique can be mitigated with the employment of feedback-direction~\cite{kn:randoop}. Effectively, the search space of possible random tests is pruned, by guiding the test generator towards valid cases, avoiding expansion on invalid test cases~\cite{kn:randoop}.

EvoSuite implements search-based automatic test generation for Java and Python code, aiming for tests that achieve high coverage, while being as small as possible and providing assertions. To achieve this they implement both evolutionary search to evolve the suite with respect to a coverage criterion and mutation testing to generate assertions~\cite{kn:evosuite}.

EvoSuiteR extends EvoSuite, aiming to provide automated generation of regression tests. Thus it takes account two versions of the software and aside from coverage, it considers state distance (``how different is the state of
all objects in the test suite across the two versions'') and control flow distance (``how
far are the two versions from diverging'')~\cite{kn:evosuiter}.

The symbolic execution technique, while introduced in the 70s has been more explored as computing power increased. It executes programs with symbolic rather than concrete values: assignments are represented as functions and conditionals as constraints. From this, we can identify specific paths and branches. This can be done statically, we symbolic execution techniques first used to derive all paths or dynamically, with symbolic execution being updated throughout execution, allowing us to find alternatives to the current branch~\cite{kn:symbolicexec}.


Similar suites have been developed, for example, for Python~\cite{kn:pynguin}.

\subsection{Test Generation with LLMs}

The recent explosion in complexity and popularity of LLMs has suscitated developer interest in their abilities to accelerate and automate software engineering. Angela Fan et al. identified that by 2023 3\% of pre-prints were related to Large Language Models and 11\% of those related to their use in software engineering~\cite{kn:angela}. Particularly relevant is their ability to generate tests, with an expectation that they could achieve better coverage, correctness and readability than previous techniques of automated test generation~\cite{kn:junjiewang}.
In comparison to traditional suites for automated test generation, such as EvoSuite, Palus, Randoop, and JTExpert, ChatGPT has shown, given right tuning of temperature settings, to evidentiate equivalent robustness~\cite{kn:gptunitbra}.
From surveys done on the topic, we find Codex and GPT variants to be the most commonly used LLMs for this issue~\cite{kn:junjiewang}.

While many studies suggest LLMs are competitive with traditional methods of software generation, we can also find evidence to the contrary. Specifically, testing across several LLMs, including Codex, StarCoder and GPT-3.5-Turbo, shows they fail in all regards compared to EvoSuite, primarily due to the generation of non-compilable code, often due to ``hallucination'' of non-existent types and methods~\cite{kn:siddiq2023empirical}. Yutian Tang and others find EvoSuite outperforms ChatGPT in code coverage~\cite{kn:tang2023chatgpt}. Given the novelty of the technology, it is unsurprising there is such variance in reported results and research, but it is worth investigating. Research on Copilot, GitHub's solution based on Codex, show interesting results: while 92.45\% of tests generated without an existing testing suite either failed, were broken or empty, by providing an existing suite the figure would drop to 54.72\%~\cite{kn:githubcopilot}, suggesting that providing this information and a structure to work off is extremely valuable, despite still wielding extremely high failure rates.

Despite variable conclusions regarding correctness and coverage, it seems generally agreed that ChatGPT is good at generating readable code. Yutian Tang et al. highlight that despite minor style errors, primarily in indentation, the generated code is clear and easy to understand~\cite{kn:tang2023chatgpt}. A survey of software developers has found ChatGPT to have comparable and even better readability than manually written tests~\cite{kn:chattester}.


Given the LLM can produce unreliable results, the maximization of ChatGPT's abilities regarding test generation has been explored: techniques such as prompting the LLM for an explanation of what the code is intending to do~\cite{kn:nuances} and feeding error messages from codes that fail to compile or execute as intended back to the LLM for correction~\cite{kn:chattester} have shown an amazing capacity for test generation, given the right prompting.

Regarding prompting, understanding how to best engineer prompts to achieve the desired output from the LLM is a particularly relevant issue. For example, in the generation of test inputs, research found ``generate diverse test input'' to be preferable over ``generate test inputs that result in different outputs between PUT and reference versions'', as ChatGPT could not accurately identify the nuances required to correctly carry out the latter's instructions~\cite{kn:nuances}.
With unit testing generation in particular, Zhiqiang Yuan et al., in developing the ChatTester tool, highlight the importance of combining a natural language descriptor with the appropriate code context~\cite{kn:chattester}.

\begin{prompt}
public class Travis {
public static final String TRAVIS NAME = ""travis-ci"";
public static final String TRAVIS = ""TRAVIS"";
public static final String TRAVIS JOB_ID = \"TRAVIS_JOB_ID\";
public static final String TRAVIS PULL_REQUEST = \"TRAVIS_PULL_REQUEST\";
public Travis (final Map<String,String> env);
public boolean isSelected();
public String getName();
public String getJobId();
public String getBranch();
public String getPull Request();

//Focal Method
public Properties getEnvironment() {
  Properties environment new Properties();
  addProperty (environment, ""travis_job_id"", getProperty (TRAVIS_JOB_ID));
  addProperty (environment, "travis_pull request", getProperty(TRAVIS PULL REQUEST));
  return environment;
  }
}

You are a professional who writes Java test methods;
Please write a test method for the method "getEnvironment()" based on the given information using Junit4.
\end{prompt}

Another technique utilized is asking the LLM to infer the intention before developing the tests, which leads it to generate correct tests where basic prompting would fail~\cite{kn:chattester}.

\begin{prompt}
Intention Prompt for ChatGPT

// Focal class
public class StrBuilder {
  public StrBuilder();

  // Focal method
  public StrBuilder setCharAt(final int index, final char c){
    if(index < 0 || index >=length()) {
      throw new StringIndexOutOfBoundsException(index);
    }
    buffer[index] = ch; return this;
  }
}

Please infer the intention of the "setCharAt(final int, final char)".
\end{prompt}

\begin{response}
The intention of 'setCharAt(final int index, final char c)' in the 'StrBuilder' class is to set the character at the specified index in the character buffer to the given character.
\end{response}

\begin{prompt}
Generation Prompt for ChatGPT

// Method Intention
The intention of 'setCharAt(final int index, final char c)' in the 'StrBuilder' class is to set the character at the specified index in the character buffer to the given character.

You are a professional who writes Java test methods.
Please write a test method for the "setCharAt(final int, final char)" with the given Method intention.
\end{prompt}

While they argue that it is little known how effective these strategies are, \citet{kn:johnny} and others highlight several prompting strategies: give examples of desired interaction, write prompts that look somewhat like code and repeat yourself. They note difficulties participants have in prompt generation, such as avoiding giving examples due to fear the LLM will simply replicate it, difficulty in searching online for help and adapting existing solutions and overgeneralizing from a single example~\cite{kn:johnny}.

In prompting we can distinguish a few types of prompts: zero-shot, where instructions are simply given, few-shot, where examples of inputs and outputs are given and few-shot with preamble, combining the previous two examples to give both a preamble instructing the LLM on what to achieve, follow by examples of inputs and outputs~\cite{kn:promptprofiannaca}

Given the natural language aspect of LLM prompting is one of the features that most distinguishes it from classical programming, it is worth exploring methods of systematizing it, making it work more like code. It is with this objective that LMQL (Language Model Query Language) was developed, introducing a scripting based query language~\cite{kn:lmql}. Evidence shows it reduces computing costs by up to 80\%~\cite{kn:lmql}.

The team behind ChatUniTest presents a two-step prompting system, where instructions of what is to be done and how are first provided~\cite{kn:chatunitest}:

\begin{prompt}
Setup Prompt
Please help me generate a JUnit test for a specific Java method using JUnit5 and Mockito3.
I will provide the source code for the method, relevant method signatures and fields, required dependencies, Java class containing the method, expected behaviour, and involved classes in the project.
Create a test that imports necessary dependencies, compiles without errors, and achieves maximum branch and line coverage. 
No explanations needed. 

Specific Test Information
The focal method is ... in the class ..., and the class information is .... The brief information of involved class ... is ...   
\end{prompt}


The training of LLMs in open source repositories gives cause to the fear that automatic test generation might be reliant on the model being trained in the code under test, or simply replicating existing testing suites. To assuage this, it was identified that even when generating tests for packages hosted in GitLab (which were not used to train the LLM under study, gpt3.5-turbo), they still maintained high coverage. In addition, it was found that 60.0\% of the tests had <= 40\% similarity to existing tests and 92.8\% had <= 50\%. Thus we can conclude the LLM is not simply copying tests from the training set without alteration~\cite{kn:max}.


\section{Summary}

The related work aids our own work in several ways: the work explored as regards unit test generation shows us a baseline functioning of the idea we are trying to implement. Particularly, we can use their statistics regarding percentage of conflicts found to assess whether our own solution is an improvement with regard to previous work.

Crucially important too is the work of \citet{kn:nuno} and, as previously mentioned, the developed Changes-Matcher.

The study of works on LLM test generation should also prove useful in the future, allowing us to get a better understanding of their functioning and capabilities, as well as providing ideas and prompt generation techniques which we may have to apply in the future.

\chapter{Empirical methodology}\label{chap:study}

The purpose of this chapter of work was to assess the capabilities of LLM to identify and test semantic conflicts.
Thus, it was divided into two phases: an initial exploratory phase, where an unstructured exploration allowed us to develop prompts, identify characteristics of the LLM and narrow down on its abilities and limitations.
In a second phase, work was systematized, with the elaboration of research questions and metrics to evaluate results.

\section{Experimental subjects}\label{chap:study:subjects}

\begin{table}[t]
\centering
% \tabcolsep=1cm
% \renewcommand{\arraystretch}{0.90}
\begin{tabular}{@{\extracolsep{\fill}} lll} \toprule
                 & Real & Conflict \\
\midrule
Point            & No  & Change Method \\
Fabricated (RO)  & No  & Remove Override \\
Fabricated (OAC) & No  & Overload by Access Change \\
Cart (CM)        & No  & Change Method \\
Cart (PF)        & No  & Parallel Field \\
Cart (PM)        & No  & Parallel Method \\
Antlr            & Yes & Parallel Field \\
OkHttp           & Yes & Parallel Method \\
\bottomrule
\end{tabular}
\caption{Subjects used in the empirical study.\label{tab:subjects}}
\end{table}

To assess the validity of a developed solution, a collection of subjects to test must be collected. While several previous work has compiled collections of merge commits with semantic conflicts, the collection done by \citet{kn:nuno} is particularly useful, being publicly available, closely related to our own work, and also allowing us to draw direct comparisons. Most importantly, it aggregates merge instances from both \citet{kn:leuson} and \citet{kn:safemerge}, while also providing valuable information, due to the work developed, such as the specific type of conflict present and whether it was detected and correctly tested by UNSETTLE (providing us with a ``base truth''). Furthermore, it has compiled a set of fabricated conflicts, which provide simpler isolated examples that can aid us as they should be easier to detect and test.
In analyzing the work of Nuno Castanho, we found the predominant types of semantic conflicts in real scenarios where ``Parallel Changes in Field'', ``Parallel Changes in Method'' and ``Change Method'' \citet{kn:nuno}. Using a simple Cart class as a base, 3 examples were made for these scenarios.
These Cart examples would be part of the group of subjects used to answer and evaluate the research questions we defined. These were complemented by a simple Point class, two examples fabricated by Nuno Castanho of type ``Override by Access Change'' and ``Remove Override'', which had achieved good results in exploratory testing, Antlr4 and OkHttp, respectively a simple and a hard to test real-world scenario.
The collection of subjects and corresponding conflict type can be found in \Cref{tab:subjects}.

\todo{Jose: Para o leitor melhor perceber os resultados (quando estes forem
apresentados), vai ter de perceber que conflictos temos aqui.  Sugiro uma breve
descricação de (1) cada programa usado e (2) uma breve descricação do conflito
presente em cada programa.}

\section{Large language models}

\todo{\begin{itemize}
  \item ChatGPT
  \item (if time allows) GitHub Copilot Chat on VS Code / IntelliJ IDE
\end{itemize}}

While preliminary work sought to explore ChatGPT and Llama (both CodeLlama and Llama 2), hardware constraints, particularly limited GPU RAM in the machines at our disposal, meant we were unable to explore Llama.  Bing AI and Bard were also considered, but they were problematic due to very stringent message size limits, in the case of Bard, and generally worse results: Bing AI, for example, could not wait for all the information to be sent, if split in more than one message and thus generated confused responses. ChatGPT, being hosted online for free and with generous message size limits, proved to be the most reliable option. Despite this, many capabalities that could prove invaluable for this work remained locked behind a premium paywall.
The instance of ChatGPT used was the web-based ChatGPT 3.5 Turbo, with default temperature, during the first semester of 2024.

\section{Research questions}

\begin{itemize}
  \item[\textbf{RQ1:}] Can ChatGPT identify, understand, and explain, whether
  there is a semantic conflict in a merge commit?

  \item[\textbf{RQ2:}] Can ChatGPT develop unit tests to identify the introduced, altered or lost behaviour due to a semantic conflict, if given an explanation of it?
\end{itemize}

\section{Experimental procedure}

\subsection{RQ1}

In our first research question we established sought to more systematically evaluate ChatGPT's capabilities to assess and describe semantic conflicts.
We first settled on a prompt, based on the prompts that were iteratively developed on previously work.

\begin{prompt}
You are a software developer that has to assess whether there is a semantic conflict in a merge commit.  Given the base version of the class, the diff from base to a version A, the diff from base to a version B and the merged version of the class, assess whether there is a semantic conflict and explain it.

Base version:
```java
```

Diff version A and the base:
```diff
```

Diff version B and the base:
```diff
```

Merge version:
```java
```
\end{prompt}

% For each semantic conflict example, as listed previously,
The prompt was submitted three times to account for the randomess nature of the
LLM, which in turn allowed us to see a broader range of responses and avoiding
what might be one-time flukes.

As an example how one would instantiate this prompt is listed below for the
Point subject, where a conflict is introduced by the simultaneous change of
euclidean to manhattan distance followed by the usage of the same distance method
in another method.  \todo{breaks when split between 3+ pages}
\begin{prompt}
You are a software developer that has to assess whether there is a semantic conflict in a merge commit.  Given the base version of the class, the diff from base to a version A, the diff from base to a version B and the merged version of the class, assess whether there is a semantic conflict and explain it. 

Base version:
```java
public class Point {
    private double x;
    private double y;

    public Point(double x, double y) {
        this.x = x;
        this.y = y;
    }

    public double getX() {
        return x;
    }

    public double getY() {
        return y;
    }

      public void move() {
          this.x += 1;
          this.y += 1;
      }

    public double distance() {
        return Math.sqrt(Math.pow(getX(), 2)+ Math.pow(getY(), 2));
    }

    public String toString() {
        return "(" + getX() + ", " + getY() + ")";
    }
}
```

Diff version A and the base:
```diff
@@ -16,8 +16,9 @@ public class Point {
      }

      public void move() {
-         this.x += 1;
-         this.y += 1;
+         double dist = distance();
+         this.x += dist;
+         this.y += dist;
      }

      public double distance() {
```

Diff version B and the base:
```diff
@@ -21,7 +21,7 @@ public class Point {
      }

      public double distance() {
-        return Math.sqrt(Math.pow(getX(), 2)+ Math.pow(getY(), 2));
+        return Math.abs(getX())+Math.abs(getY());
      }

      public String toString() {
```

Merge version:
```java
public class Point {
    private double x;
    private double y;

    public Point(double x, double y) {
        this.x = x;
        this.y = y;
    }

    public double getX() {
        return x;
    }

    public double getY() {
        return y;
    }

      public void move() {
          double dist = distance();
          this.x += dist;
          this.y += dist;
      }

    public double distance() {
        return Math.abs(getX())+Math.abs(getY());
    }

    public String toString() {
        return "(" + getX() + ", " + getY() + ")";
    }
}
```
\end{prompt}
%
% \todo{Jose: Given the motivational example is one of the subject, I might not
% sure we should provide results in here.}
% This prompt nets us the following response from the LLM.
%
% \begin{answer}
% To assess whether there is a semantic conflict in the merge commit, let's analyze the changes introduced in versions A and B compared to the base version, and then compare them with the merged version.
% 
% ### Changes introduced in Version A:
% - The `move()` method has been modified to calculate the distance using the `distance()` method and then update the `x` and `y` coordinates accordingly.
% - This change effectively moves the point by the distance it currently is from the origin.
% 
% ### Changes introduced in Version B:
% - The `distance()` method has been modified to calculate the Manhattan distance instead of the Euclidean distance.
% - The Manhattan distance is calculated by taking the absolute value of the x and y differences.
% 
% ### Analysis of Merged Version:
% - The `move()` method from Version A has been kept in the merged version, which calculates the distance using the Euclidean distance.
% - The `distance()` method from Version B has been kept, which calculates the Manhattan distance.
% 
% ### Potential Semantic Conflict:
% - The potential conflict arises from the inconsistency between how distance is calculated in the `move()` method and the `distance()` method in the merged version.
% - While the `move()` method assumes Euclidean distance for movement, the `distance()` method calculates the Manhattan distance.
% - This inconsistency could lead to unexpected behavior, as the `move()` method might not accurately reflect the intended behavior of the `distance()` method.
% 
% ### Conclusion:
% - There is a semantic conflict in the merge commit because the merged version contains methods with conflicting assumptions about how distance should be calculated.
% - Resolving this conflict would require ensuring consistency in the distance calculation method used throughout the class, either sticking with Euclidean distance or Manhattan distance for both `move()` and `distance()` methods.
% \end{answer}

To evaluate the answers generated by ChatGPT, we established five metrics, with likert scales:
%
\begin{itemize}
  \item[\textbf{A}:] Changes between branches correctly identified (Correct description of Version A and B).
  \begin{itemize}
    \item[\textbf{0}:] Large errors in description for both branches.
    \item[\textbf{1}:] Large error in description of one branch.
    \item[\textbf{2}:] Minor insignificant errors.
    \item[\textbf{3}:] Changes correctly identified.
  \end{itemize}

  \item[\textbf{B}:] No conflict misunderstanding (Does not describe textual conflict, understand merge succeed).
  \begin{itemize}
    \item[\textbf{0}:] Understands conflict as textual.
    \item[\textbf{1}:] Understands the conflict as semantic.
  \end{itemize}

  \item[\textbf{C}:] Positive response (3 types: says conflict exists/may exist/does not exist)
  \begin{itemize}
    \item[\textbf{0}:] Denies existence of conflict
    \item[\textbf{1}:] Asserts conflict is possible
    \item[\textbf{2}:] Asserts conflict exists
  \end{itemize}

  \item[\textbf{D}:] Origin of conflict described (What code interactions lead to altered behaviour)
  \begin{itemize}
    \item[\textbf{0}:] Incorrect or non-existent explanation
    \item[\textbf{1}:] Identifies origin of conflict, with lack of clarity or imprecisions
    \item[\textbf{2}:] Identifies origin of conflict, with lack of confidence
    \item[\textbf{3}:] Identifies origin of conflict
  \end{itemize}

  \item[\textbf{E}:] Effect of conflict described (What is the result of the code interactions/expected output)
  \begin{itemize}
    \item[\textbf{0}:] Result of conflict is omitted, too vague or wrong
    \item[\textbf{1}:] Result of conflict is expressed, but with imprecisions or generically
    \item[\textbf{2}:] Possible code outputs are expressed, with little confidence
    \item[\textbf{3}:] Assertively points out expected outputs due to conflict
  \end{itemize}
\end{itemize}

% From this we derive the metrics: A=3, B=1, C=2, D=3, and E=2.
% In this case, given the correctness of the answer, all metrics have perfect scores,
% except for metric D (Effect of conflict described), due to the reduced confidence.

\subsection{RQ2}

Research Question 2 sought to evaluate chatGPT's test generation ability. Specifically, whether it could generate tests that would show the evidence the semantic conflict, given an explanation.
For example, taking the Point, if ChatGPT generates a unit test that tests the move function, we would identify the conflict if the test fails in merge (assuming the test asserts Euclidean based movement as true) or in branch B (assuming it asserts Manhattan based movement as true).
We evaluate by observing whether a functional unit test is generated and if not, the edit distance to a functioning version of one test.

\section{Threats to validity}

\todo{Describe any threat to our study, e.g., subjects, llms, our prompts, and
what have we done to mitigate them.}

Based on the guidelines reported by \citet{wohlin2012experimentation}, we have
taken all reasonable steps to mitigate the effect of potential threats, which
are described in detail in this section.

\subsection{Threats to construct validity}
%
\todo{Are associated with the correspondence between theory and observation.}

\subsection{Threats to internal validity}
%
\todo{Are associated with uncontrollable internal factors that may influence our results.}

\subsection{Threats to external validity}
%
\todo{Are associated with the generalization of the results reported.}

    \chapter{Experimental results}\label{chap:results}

This chapters answers \textbf{RQ1 – RQ3}, which study ChatGPT's ability to identify and generate tests for semantic conflict, the latter
with both our prompt, which provides an explanation, and state-of-the-art prompts, which don't.

\section{RQ1}\label{sec:results:rq1}

\begin{table}[t]
\centering
% \tabcolsep=1cm
% \renewcommand{\arraystretch}{0.90}
\begin{tabular}{@{\extracolsep{\fill}} lrrrrr} \toprule
                 & A [0-3] & B [0-1] & C [0-2] & D [0-3] & E [0-3] \\
\midrule
Point (CM)       & 3.0 & 1.0 & 0.7 & 1.0 & 0.7 \\
Fabricated (RO)  & 3.0 & 1.0 & 1.0 & 2.0 & 0.7 \\
Fabricated (OAC) & 2.3 & 1.0 & 1.0 & 3.0 & 0.7 \\
Cart (CM)        & 3.0 & 1.0 & 2.0 & 3.0 & 2.7 \\
Cart (PF)        & 2.3 & 1.0 & 0.0 & 0.0 & 0.0 \\
Cart (PM)        & 2.0 & 1.0 & 1.3 & 1.0 & 0.3 \\
Antlr (PF)       & 1.7 & 1.0 & 0.0 & 0.0 & 0.0 \\
OkHttp (PM)      & 3.0 & 1.0 & 0.7 & 0.0 & 0.0 \\
Retrofit (PM/CM) & 1.0 & 1.0 & 1.0 & 0.0 & 0.0 \\
\midrule
\textit{Average}  & 2.4 & 1.0 & 0.9 & 1.1 & 0.6 \\
\bottomrule
\end{tabular}
\caption{Average values per subject and metric.\label{tab:results:rq1}}
\end{table}

\Cref{tab:results:rq1} summarizes the metrics achieved per subject. From these we can note that:

\begin{itemize}
  \item \textbf{ChatGPT is generally able to identify changes made between branches} ($A=2.4$).

  \item \textbf{In none of the subject did the tool mistake the kind of conflict under discussion} ($B=1.0$). This might be because the prompt did not assert the existence of a conflict, allowing negative responses, whereas other prompts, which asked to find a conflict that did exist, lead the model to find any explanation, which often consisted of describing a textual conflict.

  \item \textbf{Conflict assertion remains uncertain}. ($C=0.9$) In many cases, particularly those of PF type, no conflict as identified. In many other cases it was found but with uncertainty, e.g., with words such as ``possible'', and ``might''.

  \item \textbf{On average, it can identify what originates conflicts} ($D=1.1$).  Despite the range given, values given to the metric were either 0 or 3. This indicates to us in nearly half the cases, the model failed to produce any accurate description, but when it did, it was generally very accurate.

  \item \textbf{ChatGPT struggles with accurately describing the results of semantic conflicts} ($E=0.8$), i.e., how the behaviour is specifically altered and how the outputs change. Indeed, even when the origin of the conflict is found, an accurate description of the results does not necessarily follow, with the topic being vague or not touched upon.
\end{itemize}

The complete prompts, with results and evaluated metrics, can be found in the attachments.
An important observation to make is the variability of results, which highlights the importance of running three trials for our evaluation.
For example, the point class had high variation: two of the results failed to identify any conflict at all, but one of the three not only identified, but quite accurately describe its origin and results.

In the more complex fabricated examples, the biggest issue was the identification of the effects of the semantic conflict, namely that when an overriden function was removed or had its access changed, the other function, with different behaviour, would be used instead. However, given the constraints of the prompt, which is limited to one class, the necessary information could not be transmitted and the LLM argued there would be compilation errors. 
Better results were observed when doing more manual, ad-hoc experiments, as we could identify exactly all the classes involved in the conflict and offer all the required information.

As we proceeded testing the ``Cart'' scenario, we observed a reversal of a trend of very low results for the E metric, particularly for CM and PM.
Indeed for these examples, not only did the LLM identify the origins of the conflict, but exactly the result of this, such as:
``Now, in the merged version, both changes are included. So, when an admin user proceeds to checkout (assuming the admin check in version A evaluates to true), the cost will be set to 0. But immediately after, the checkout method checks if the cost is 0 and throws an exception if it is. This leads to a contradictory behavior: the system sets the cost to 0 for admin users but then doesn't allow them to proceed with checkout due to the cost being 0.''
It is possible this is due to the very straighforward logic of the CM example in particular. What is surprising is that due to some minor changes in the Cart class, we re-ran these tests and originally these values were much lower. Why there was such an improvement is unknown but it is evidence of the variability of these tools.

In a scenario where detection failed we see an example of the LLM providing a grossly wrong explanation: ``If the user is an admin and a discount code that results in a cost of 0 is applied, the exception will never be thrown because the `checkout` method will not be called.''. As regards the conflict itself, this was only completely wrong assertion of its kind: in general where answers failed they were either vague or omitted, not present. The only other major example of a factually incorrect answer came with OkHttp:
\begin{lstlisting}
### Explanation:
The conflict arises because both changes affect the same field (`hostnameVerifier`), but they prescribe different default values.
    
In the merge commit, the decision was made to adopt the change from Version B, which explicitly sets the default `hostnameVerifier` to `OkHostnameVerifier()`. This decision potentially overrides any behavior or configuration related to the default `hostnameVerifier` set in Version A.
    
If the intended behavior was to preserve the default `hostnameVerifier` as defined in Version A, then this would be considered a semantic conflict, as the behavior of the merged version may differ from the behavior of either Version A or Version B individually.
\end{lstlisting}
In this case, the changes made in only one branch are described as a conflict.

Another case of wrong information given by ChatGPT regarded the changes made, both in Cart(PF), which 
correctly describes the removal but incorrectly counts:
\begin{lstlisting}
    1. **Diff Version A**:
    - It removes two discount codes, `"superpromo"` and `"winner"`.
    - This version retains 6 discount codes.
 
 2. **Diff Version B**:
    - It removes one discount code, `"winner"`.
    - This version also retains 6 discount codes.
\end{lstlisting}
As well as Antlr, which fails to accurately describe the changes made:
\begin{lstlisting}
1. **Diff A vs Base:**
    - In version A, the `reduce` keyword was replaced with `return` in the `python2Keywords` array.
 
2. **Diff B vs Base:**
    - In version B, the `del` keyword was added to the `python2Keywords` array.
-------------------------------------------------------------------------------
1. **Diff A vs. Base**:
    - Addition of `"return"` keyword in the `python2Keywords` array.
    - No conflicting changes with other methods or properties.
 
2. **Diff B vs. Base**:
    - Replacement of `"delattr"` with `"del"` in the `python2Keywords` array.
    - No conflicting changes with other methods or properties.
\end{lstlisting}
We observed both of these are ``Parallel Field'' semantic conflicts so it is possible the fact that the changes are being made to fields may have some relation to its inability to express them correctly. 
Further credence to this involves Cart (PM), where edits to the field of an auxiliary class also were misdescribed.
In contrast stands Retrofit: while ChatGPT reliably failed to correctly describe the changes, no edits were made to fields. From observation, we can say these errors may be due
to the fact that the changes are similar and repeated across two complex methods, for branch B. We still see major misdescriptions for branch A, which is relatively simple, with a simple addition of an append to a string.
We see examples of aspects of branch B being assigned to it, and then those same aspects not acknowledged for branch B.
\begin{lstlisting}
    1. In `buildRelativeUrl()`:
    - Version A adds a null check for path parameters and throws an `IllegalArgumentException` if any of them are null. This ensures that path parameters are not null, which could lead to unexpected behavior or errors.
    - Version B also adds a null check for path parameters but handles it differently. Instead of throwing an exception, it continues execution. This could lead to inconsistencies if the application logic relies on path parameters being non-null.
 
 2. In `buildBody()`:
    - Both versions A and B have added null checks for the body parameter in the `SIMPLE` case. This ensures that the body is not null before proceeding with further processing. However, the behavior is consistent between the two versions. 
\end{lstlisting}

Like with Antlr and Retrofit, OkHttp generated no conclusive answers. In particular it tended to offer advice on how identify and diagnose semantic conflicts and make good merges. This is a behaviour that was also observed with other complex, real-world conflicts, during the period of ad-hoc testing.
\begin{lstlisting}
### Potential Semantic Conflict:
While there is no direct semantic conflict, there could be conflicts in the usage and interpretation of the `transports` field and its associated methods. For instance:
- How are these transports selected and utilized within the class?
- Do they introduce any performance or compatibility issues?
- How do these changes impact existing functionality and client behavior?
---------------------------------------------------------------------------------------
### Semantic Conflict Assessment:
Looking at the changes introduced in versions A and B, there does not seem to be a direct semantic conflict between them. Version A primarily deals with transport configuration and introduces a new field `transports` along with related methods. On the other hand, version B modifies the hostname verifier behavior.

However, when merging both changes, we need to ensure that the introduced hostname verifier (`OkHostnameVerifier`) is consistent with the overall configuration of the `OkHttpClient`. Specifically, we need to verify whether `OkHostnameVerifier` is compatible with the behavior and expectations of the existing codebase.
\end{lstlisting}

Evaluating the results as a whole, it is worth imagining in which way the types of conflicts affect the model's capabilities. 
As observed earlier, it seem that Parallel Field conflicts induce errors in branch change description. They also fail to identify the conflict but this seems unrelated to failures of both description:
both cases with succesful and failing descriptions failed to produce conflict identification. Indeed, this was the case, we would expect some description of a conflict, with the same imprecisions, whereas
for our PF cases no conflict was even considered to exist. Our hypothesis is that the conflict is far too simple to be detected: in both cases some type of collection is altered, either with insertions or removals.
The merging of these alterations leads to emergent behaviour: the sets and lists returned are different, with different sizes, when compared to both branches.
However while the behaviour is emergent, it is both expected and likely desirable, hence why it is probably not being seen as conflict.

It is difficult to assess to what extent other types of conflicts affect detection ability. If we observe PM and CM, Cart (PM) and Point (CM) are comparable, while Cart (CM) has near perfect scores and OkHttp (PM) very low ones.
It seems by far that the complexity of the conflict is the determining factor and this complexity can be noted as the complexity of the whole class under test (with OkHttp being the clear outlier), but also in the intricacies of the conflict.
We observe Cart (CM) is fairly straighforward: if the user is an admin, checkout throws and exception. Indeed, even without any of the merge info, it would be fairly trivial to identify this is unintended, whereas
Point and Cart (PM) require the identification of the original behaviour and how it has changed.

It is also worth pointing out that despite good results for metric D in Remove Override and Overload by Access Change, with a perfect 3 in the latter, explanations of the result was flawed, due to a lack of information, ChatGPT reliably understood that trying a function that was removed or that had its access changed would produce errors. Thus it is worth noting in this case, accurate identification of the conflict's result would require a prompt that provides further information.

Overall, we can say ChatGPT can indeed identify and explain whether there is a semantic conflict in a merge. This comes with many caveats however, firstly being that identification does not necessarily lead to a proper explanation. This is not a particularly pressing issue, as a tool that identifies conflicts and then alerts humans, who can themselves describe and fix the issue would already be a great boon for efficient software development.

The bigger caveat is the situations in which it identifies the conflicts: simple fabricated scenarios with significant alterations to behaviour. When the emergent behaviour is small and expected, such as those of PF, the LLM does not view it as a conflict, indicating a bias to clearly unexpected and unintended behaviour. On the other side, if the class is of significant complexity, it becomes harder for the LLM to extract valuable information.
We see these issues combined in OkHttp as the conflict is straighforward: parallel changes result in an object that is different from both branches, thus emergent. Simultaneously, the class is significantly more complex than the other subjects. Thus while in the PF examples, the possibility of a conflict is flat out dismissed,
two of the answers for OkHttp accept the possibility, arguing that more study needs to be done on transports and the hostname verifier respectively.

Further work should thus tackle this: on one hand exploring how to provide a better understanding of semantic conflicts, specifically that emergent behaviour is the result of semantic conflict, even if it does not introduce an apparent bug, software vault or unexpected behaviour.
On the other hand, real-world scenario with complex dependencies, large classes with varied functions, inheritance and other features complicates detection and more complex prompts, data extraction and presentation are likely required.

\section{RQ2}\label{sec:results:rq2}

\Cref{tab:results:rq2} summarizes the results achieved, split by conflict and evaluation type.
These list the passing tests, after edits, the average edit distance to tests that compile and pass in merge 
and the errors present that stop the conflict from being detected.

\Cref{tab:results:rq2tests} gives the values for the tests, after compilation errors are fixed in merge.
These come in the format Base A B Merge. C stands for compilation failures, R for runtime errors, F for failing test and P for passing tests.
In some cases there are several listing per trial, due to the existence of several unit tests.
The first two results for Cart (CM) ChatGPT are italicized: these utilized deprecated features of Junit, and required rolling back to an earlier
version to run.

OkHttp is omitted from the results, as the prompt caused it to only generate imports. It is unknown why.

From this we can observe:
-The main obstacle to the detection of the conflict was incorrect asserts, present in nearly every case.
-In non-compiling tests, those generated with explanations given by ChatGPT tend to require more correction.
-Tests are generally compilable or very close to succesful compilation, with constructor and other errors, such as usage of non-existent getters as the main issues.

\begin{table}[t]
    \centering
    \begin{tabular}{@{\extracolsep{\fill}} lrrr} \toprule
                             & Successes (of 3) & Average Edit Distance & Errors \\
    \midrule
    Point (CM) Manual        & 3 & 4 & Assert \\
    Point (CM) ChatGPT       & 3 & 4 & Assert \\
    Fabricated (OAC) Manual  & 3 & 2 & Assert/Branch/Other \\
    Fabricated (OAC) ChatGPT & 3 & 4 & Assert/Branch \\
    Fabricated (RO) Manual   & 1 & 3 & Assert/Miss \\
    Fabricated (RO) ChatGPT  & 1 & 3 & Assert/Miss \\
    Cart (CM) Manual         & 3 & 68 & Assert/Constructor \\
    Cart (CM) ChatGPT        & 3 & 34 & Constructor/Other \\
    Cart (PF) Manual         & 3 & 2 & Assert \\
    Cart (PM) Manual         & 0 & 18 & Other \\
    Cart (PM) ChatGPT        & 1 & 42 & Assert/Other \\
    Antlr (PF) Manual        & 3 & 1 & Assert \\
    Retrofit (PM/CM) Manual  & 0 & N/A & Other \\
    \midrule
    \end{tabular}
    \caption{Results by conflict and explanation type.\label{tab:results:rq2}}
\end{table}

\begin{table}[t]
    \centering
    \begin{tabular}{@{\extracolsep{\fill}} llcccccccccccccc} 
        \toprule
        Prompt & & \multicolumn{14}{c}{Trial} \\ 
        \cmidrule(lr){3-16}
        & & \multicolumn{4}{c}{1} & \multicolumn{6}{c}{2} & \multicolumn{4}{c}{3} \\ 
                            & & b & A & B & M & & b & A & B & M & & b & A & B & M \\ 
\midrule
Point (CM) Manual           & & F & F & F & P & & F & F & F & P & & F & F & F & P \\ 
Point (CM) ChatGPT          & & F & F & F & P & & F & F & F & P & & F & F & F & P \\ 
                            & & F & F & P & P & & F & F & P & P & & F & F & P & P \\ 
Fabricated (OAC) Manual     & & C & C & F & P & & C & C & F & P & & C & C & F & P \\ 
Fabricated (OAC) ChatGPT    & & C & C & F & P & & C & C & F & P & & C & C & F & P \\ 
Fabricated (RO) Manual      & & C & F & C & P & & C & P & C & P & & C & P & C & P \\ 
Fabricated (RO) ChatGPT     & & C & F & C & P & & C & P & C & P & & C & P & C & P \\
Cart (CM) Manual            & & F & F & F & P & & F & F & F & P & & F & F & F & P \\ 
                            & &   &   &   &   & & P & P & P & P & &   &   &   &   \\ 
Cart (CM) ChatGPT           & & \textit{F} & \textit{F} & \textit{F} & \textit{P} & & \textit{F} & \textit{F} & \textit{F} & \textit{P} & & F & F & F & P \\ %italicize first two
                            & &   &   &   &   & &   &   &   &   & & F & F & P & P \\ 
Cart (PF) Manual            & & F & F & F & P & & F & F & F & P & & F & F & F & P \\ 
                            & & F & F & F & P & & F & F & F & P & & F & F & F & P \\ 
Cart (PM) Manual            & & C & C & P & P & & C & C & P & P & & C & C & P & P \\ 
                            & & P & P & P & P & & C & C & P & P & & P & P & P & P \\ 
                            & &   &   &   &   & & C & C & P & P & & C & C & P & P \\ 
                            & &   &   &   &   & & C & C & P & P & & P & P & P & P \\ 
Cart (PM) ChatGPT           & & C & C & P & P & & C & C & P & P & & C & C & P & P \\
                            & & C & C & P & P & & C & C & F & P & & P & P & P & P \\
                            & &   &   &   &   & & C & C & P & P & & C & C & P & P \\
                            & &   &   &   &   & & C & C & F & P & & P & P & P & P \\
Antlr (PF) Manual           & & F & F & F & P & & F & F & F & P & & F & F & F & P \\ 
        \bottomrule
    \end{tabular}
    \caption{Test running results in base A B Merge \label{tab:results:rq2tests}}
\end{table}


Analysis of \Cref{tab:results:rq2tests} allows us to identify which cases identify the conflict, and which did not.
It is worth examining the results and what they mean, starting by the failing examples:

-PPPP: The test always passes, due to the correct testing of a feature not affected by the merge.

-FFPP and CCPP: The test passes in branch B and in merge. This indicates a behaviour was introduced and remained unaltered in the merge, thus it is not a conflict in itself.
This was common for Point (CM) ChatGPT, as the prompt induced the generation of tests for distance(), which correctly calculated and asserted the euclidean distance of a given point.
They however failed to do the same when applying the distance to move(), thus those required correction.

-CPCP: The same as the previous example, but the change introduced is in branch A.

The examples that show a semantic conflict is present require failure in all brances except merge, or compilation failure in base and one of the branches and failure on the other.

These are:

-FFFP
-CCFP
-CFCP

These all show that behaviour is emergent: something which was never true in any of the previous branches becomes true. The behaviour being asserted is thus the 
emergent behaviour that occurs in merge.

If we observe the results, particularly the errors found, we can see that the main issue present
is incorrect assertions. While in some cases these are understandable (such as the case of RO,
where the values to verify are hash and thus hard to know without running the code), most of the time
they were the result of simple logical or mathematical errors, things that are trivial for a human to get right
and thus also correct.
It is hard to identify why exactly it finds it so hard to get the right asserts. If we observe the cart example,
we find that tests generated for distance() were always correct. These consisted of a simple x + y addition.
However, tests generated for move() were always wrong; in turn these were just x + distance() and y + distance() additions.
Given that it shows it can correctly calculate distance() it is hard to fathom why it struggles
to use it in addition: we can perhaps suggest there is making the association between the method call and the returned value it represents.

Ocasional compilation errors occurred due to the generation of methods that are not present in the code, such as getUser().
In one case, we find an interesting error: the codebase contains methods named ```setResgistration'''.
For the most part they are ignored, as they are not necessary to identify the conflict. In one trial, however,
the method is called for Producer. However the ```Resgistration''' typo is corrected: ``producer.setRegistrationNumber(123);''.
Naturally, this stops code from compiling, but it is a noteworthy example of how Large-Language Models, due to their nature as 
predictive models, can fail in outlier cases, such as an unexpected typo in method namings.

It is noticeable that for Cart(CM), the manual explanation test cases required relatively more edit, while the reverse
is observed for Cart(PM). An explanation might be that one type of prompt generates more complex and longer tests. However as can be seen in
\cref{tab:results:rq2tests}, both examples have comparable amount of test cases and test length does not vary much.
It is possible these are the results of outliers: that if we for example observe Cart (PM), ChatGPT explanation, the results are significantly weighed by
trial 3, which requires significant correction due to the way it tries to access the user:
\begin{answer}
    @Test
    public void testSumItemsWithCoupon() {
        Cart cart = new Cart(new User());
        Item item = new Item(50, true); // Item with coupon
        cart.add_item(item);
        assertEquals(-50, cart.sum_items(), 0.001);
    }

    @Test
    public void testSumItemsWithoutCoupon() {
        Cart cart = new Cart(new User());
        Item item = new Item(50); // Item without coupon
        cart.add_item(item);
        assertEquals(50, cart.sum_items(), 0.001);
    }

    @Test
    public void testTotalCostWithCoupon() {
        Cart cart = new Cart(new User());
        Item item = new Item(50, true); // Item with coupon
        cart.add_item(item);
        cart.checkout("loyal");
        assertEquals(-45, cart.getUser().getBalance(), 0.001);
    }

    @Test
    public void testTotalCostWithoutCoupon() {
        Cart cart = new Cart(new User());
        Item item = new Item(50); // Item without coupon
        cart.add_item(item);
        cart.checkout("loyal");
        assertEquals(-45, cart.getUser().getBalance(), 0.001);
    }
\end{answer}
has an edit distance of 121, has correction requires extracting the user for access:
\begin{answer}
    @Test
    public void testSumItemsWithCoupon() {
        Cart cart = new Cart(new User());
        Item item = new Item(50, true); // Item with coupon
        cart.add_item(item);
        assertEquals(0, cart.sum_items(), 0.001);
    }

    @Test
    public void testSumItemsWithoutCoupon() {
        Cart cart = new Cart(new User());
        Item item = new Item(50); // Item without coupon
        cart.add_item(item);
        assertEquals(0, cart.sum_items(), 0.001);
    }

    @Test
    public void testTotalCostWithCoupon() {
        User user = new User();
        Cart cart = new Cart(user);
        Item item = new Item(50, true); // Item with coupon
        cart.add_item(item);
        cart.checkout("loyal");
        assertEquals(0, user.balance, 0.001);
    }

    @Test
    public void testTotalCostWithoutCoupon() {
        User user = new User();
        Cart cart = new Cart(user);
        Item item = new Item(50); // Item without coupon
        cart.add_item(item);
        cart.checkout("loyal");
        assertEquals(0, user.balance, 0.001);
    }
\end{answer}
This same error of trying to access user from cart can be found in other ChatGPT results in PM, but it is ambiguous
whether this is somehow the result of the prompt or coincidental generations. On the other side, Cart (CM) tended to generate
asserts of type AssertThrow and AssertNotThrow, which often had to be reversed to be in line with merge, for manual, whereas for
ChatGPT explanation the conflict these asserts were more often correct. This is clearly an issue with the manually written explanation,
which claims the error throwing behaviour in merge is not intended, leading for tests that assert that nothing is thrown.
It is thus likely good practice to not make value judgements on the behaviour of the code in merge.

On the topic of compilation errors, it is worth nothing often, for Cart(CM) these were related to constructors, as the constructors
for the other classes involved in the Cart conflicts (Item and User) were not included in the prompt. It is interesting
to observe what the common error is: the introduction of extra parameters. For example, in the code of Cart only the fields
value and amount of Item are accessed. However, Item was frequently iniatilized with a constructor with 3 parameter: Name,
Value and Amount. It is fair to assume the Item class would both have a name field and that it would come first in the constructor
and it is likely the training data had many examples of this. Thus it is worth pointing out that testability of code by LLM's can be
improved by closely following norms and rule of thumb expectations and corollarily LLMs will struggle to effectively construct tests
for more unusual pieces of software.

A way to resolve this issue would be to include constructor headers in the original prompt. However one issue is the selection of which
headers: if instead of just Item and User, the Cart class had 20 other dependencies which bore no relevance in our conflict, we would be 
providing extra information in the prompt which could cause confusion. A possibly better option is feeding compilation messages back for correction,
as shown by ChatTester~\cite{kn:chattester}, possibly with the addition of corresponding headers. This corrective approach could also be utilized
to fix assert errors. In effect, it would be an automated way to do the manual corrections we carried out.

Some other issues with constructors we observed are not so obvious to solve: namely that despite knowing of available construcotrs, there's no guarantee
the correct one for the situation will be used. We can see the case of Cart (PM): due to changes in the Item class, constructor information came included
in the diff section of the prompt. This led to the creation of tests using (Value, Coupon) and (Value) format.
\begin{answer}
    public class CartTest {

    @Test
    public void testSumItemsWithCoupon() {
        Cart cart = new Cart(new User());
        Item item = new Item(50, true); // Item with coupon
        cart.add_item(item);
        assertEquals(-50, cart.sum_items(), 0.001);
    }

    @Test
    public void testSumItemsWithoutCoupon() {
        Cart cart = new Cart(new User());
        Item item = new Item(50); // Item without coupon
        cart.add_item(item);
        assertEquals(50, cart.sum_items(), 0.001);
    }

    @Test
    public void testTotalCostWithCoupon() {
        Cart cart = new Cart(new User());
        Item item = new Item(50, true); // Item with coupon
        cart.add_item(item);
        cart.checkout("loyal");
        assertEquals(-45, cart.getUser().getBalance(), 0.001);
    }

    @Test
    public void testTotalCostWithoutCoupon() {
        Cart cart = new Cart(new User());
        Item item = new Item(50); // Item without coupon
        cart.add_item(item);
        cart.checkout("loyal");
        assertEquals(-45, cart.getUser().getBalance(), 0.001);
    }
}
\end{answer}
Notably these constructors don't set the amount field, which is thus 0, rendering the items useless for any value calculations.
Rectification would require understanding this logic error and thus is liable to be more difficult to achieve. It is for these representations
we have so few successes for Cart (PM): while we corrected the asserts to get passing tests in merge (in this case, setting them to 0, as no
value is subtracted), these corrections don't aid in identifying the conflict.

The other case where the conflict is rarely identified despite our corrections is RO, as it often generated asserts which always passed:
\begin{answer}
    @Test
    public void testHashString() {
        InstalacaoAssentos instalacao = new InstalacaoAssentos();
        assertNotNull(instalacao.hashString());
    }
\end{answer}
Within the parameters of our work and a corrective system that works off compilation and test errors, this is the most problematic case: as the test works perfectly in merge,
there is nothing prompting us for repair and repair would require more than simply changing a value, but to actually change the type of assert itself.

Retrofit stands as an outlier and indeed we did correct the generated tests as they are so deviated from correctness, any edit would have to fundamentally alter the setup and structure of the test,
to a point where if henceforth the test worked, we could not say it reflected anything on ChatGPT's capabilities. 
From observing, it is clear the LLM in a way does not have enough information to correctly generate tests and thus opts for generating non-existent mocks:
\begin{answer}
    @Test(expected = IllegalArgumentException.class)
    public void testBuildRelativeUrlWithNullPathParam() throws UnsupportedEncodingException {
      RequestBuilder builder = new RequestBuilder(mockConverter());
      builder.methodInfo(mockMethodInfo());
      builder.apiUrl("http://example.com");
      builder.args(new Object[]{null}); // Pass null argument
      builder.build(); // This should throw IllegalArgumentException
    }
\end{answer}
As well as extensive incomplete test suites, which end up being almost all the same.
\begin{answer}
    @Test(expected = IllegalArgumentException.class)
    public void testBuildRelativeUrl_NullPathParam() throws UnsupportedEncodingException {
      RequestBuilder builder = new RequestBuilder(mock(Converter.class));
      builder.methodInfo(mock(RestMethodInfo.class))
             .apiUrl("http://example.com")
             .args(new Object[] {null})
             .build();
    }
  
    @Test(expected = IllegalArgumentException.class)
    public void testBuildBody_NullBody() {
      RequestBuilder builder = new RequestBuilder(mock(Converter.class));
      builder.methodInfo(mock(RestMethodInfo.class))
             .apiUrl("http://example.com")
             .args(new Object[] {null})
             .build();
    }
  
    @Test(expected = IllegalArgumentException.class)
    public void testBuildBody_NullFormUrlEncodedField() {
      RequestBuilder builder = new RequestBuilder(mock(Converter.class));
      builder.methodInfo(mock(RestMethodInfo.class))
             .apiUrl("http://example.com")
             .args(new Object[] {null})
             .build();
    }
  
    @Test(expected = IllegalArgumentException.class)
    public void testBuildBody_NullMultipartPart() {
      RequestBuilder builder = new RequestBuilder(mock(Converter.class));
      builder.methodInfo(mock(RestMethodInfo.class))
             .apiUrl("http://example.com")
             .args(new Object[] {null})
             .build();
    }
\end{answer}
It is worth noting none of the tests seem to attempt to test both modified behaviours at once (? addition to query parameters and null handlings). Thus even if correctly designed, they would not detect the conflict.
However this could always be a prompt issue.

It is worth nothing Retrofit has a large existing test suite, including a Helper class which is very helpful for setting up the RequestBuilder with appropriate arguments.
Exploratory work had shown that feeding this test class and prompting for generation on its style could lead to the generation of working (but not detecting) tests. It is probable then,
that for these complex real-world scenarions, aiding test generation by giving examples of existing test suites will be a necessary part of the prompt. Indeed, even if ChatGPT could generate
working tests regardless, it would still be useful for developers to have them more closely conform the existing suite.

Despite all the problems mentioned previously, the results are generally quite good, as most errors are relatively small and easily fixable with further prompting or manual edits as we did.
Thus we can say with some certainty that for these simple cases, most of which where ChatGPT can identify the conflict with some degree of confidence, it can also generate appropriate unit tests.
It is unfortunate that test generation for OkHttp failed so badly was to not even generate tests, as that class was far more intricate and could give evidence to issues such as dealing with
private methods. But it has to be understood that these sort of issues are a possibility given the volatility of LLMs.

\section{RQ3}\label{sec:results:rq3}


\begin{table}[t]
    \centering
    \begin{tabular}{@{\extracolsep{\fill}} lrr} \toprule
                     & Test Results & Possible Identification \\
    \midrule
    Point (CM)       & Passing & No \\
    Fabricated (RO)  & Failing & Yes \\
    Fabricated (OAC) & Non-Compiling & Yes \\
    Cart (CM)        & Non-Compiling & No \\
    Cart (PF)        & Passing & Yes \\
    Cart (PM)        & Non-Compiling & Yes \\
    Antlr (PF)       & Non-Compiling & No \\
    OkHttp (PM)      & Non-Compiling & No \\
    Retrofit (PM/CM) & Non-Compiling & No \\
    %\midrule
    \bottomrule
    \end{tabular}
    \caption{Results for Prompt 1.\label{tab:results:rq3p1}}
    \end{table}

\begin{table}[t]
    \centering
    \begin{tabular}{@{\extracolsep{\fill}} lrr} \toprule
                     & Test Results & Possible Identification \\
    \midrule
    Point (CM)       & Failing & Yes \\
    Fabricated (RO)  & Passing & No \\
    Fabricated (OAC) & Failing & Yes \\
    Cart (CM)        & Non-Compiling & Yes \\
    Cart (PF)        & Non-Compiling & No \\
    Cart (PM)        & Non-Compiling & Yes \\
    Antlr (PF)       & Non-Compiling & No \\
    OkHttp (PM)      & Non-Compiling & No \\
    Retrofit (PM/CM) & Non-Compiling & No \\
    %\midrule
    \bottomrule
    \end{tabular}
    \caption{Results for Prompt 2.\label{tab:results:rq3p2}}
    \end{table}

\begin{table}[t]
    \centering
    \begin{tabular}{@{\extracolsep{\fill}} lrr} \toprule
                     & Test Results & Possible Identification \\
    \midrule
    Point (CM)       & Runtime & No \\
    Fabricated (RO)  & Failing & No \\
    Fabricated (OAC) & Non-Compiling & No \\
    Cart (CM)        & Non-Compiling & No \\
    Cart (PF)        & Passing & Yes \\
    Cart (PM)        & Non-Compiling & Yes \\
    Antlr (PF)       & Non-Compiling & Yes \\
    OkHttp (PM)      & Non-Compiling & Yes \\
    Retrofit (PM/CM) & Non-Compiling & No \\
    %\midrule
    \bottomrule
    \end{tabular}
    \caption{Results for Prompt 3.\label{tab:results:rq3p3}}
    \end{table}

--still quite good results 

--higher randomess (PF testing for size, Cart CM testing with admin)

--randomness offset but more tests, but more corrective work then

--in prompt repair framework, not only edit distance to consider edit amount
(ie many tests will require many more promptings, which translates to time and
money)

--prompts with only method signatures are naturally worse at CM type conflicts

--excessive mocking in RQ3
\chapter{Conclusion \& Future Work}\label{chap:conclusion}

\section{Conclusion}

In our preparation for dissertation we explored and presented the problem of software merges, more specifically, the difficulty in establishing an automated method to detect, identify and highlight semantic conflicts that arise.

We explored related work that can provide avenues of study and structures to guide the development of our solution. Specifically, previous work on test generation to identify semantic conflicts, the state of the art of automated test generation and the usage of Large Language Models for test generation, with specific focus on the important of appropriate prompting as well as output correction.

By establishing the three research question we established, we could investigate the viability of using LLMs in this way. From this we can say:

- Research Question 1: We established that ChatGPT has some capability to identify semantic conflicts. One of the obstacles found was understanding:
conflicts which did not create obvious code faults (such as parallel addition to lists) were not seen as conflicts. 

- Research Question 2: We found that despite many small errors with the test setup and asserts, when an appropriate explanation was given, ChatGPT could generate
tests that appropriately verified the required behaviours. We also found no significant difference between human written and ChatGPT written explanations. The main obstacle
was the complexity of some subjects, particularly the real-world ones, where ChatGPT struggled to generate adequate tests.

- Research Question 3: By investigating state of the art prompts, we observed that while they could still generate tests that adequately tested the conflict (after correction),
these came with downsides, due to the lack of focus provided by the detailed explanation: there was higher variability as it was not guaranteed the right setup would be attempted.
For example, in tests that required a User class to have the admin field set to true, nothing in the prompt indicated this and while generating many tests will increase the chances
of this field being set, more tests necessarily imply more time and work.

Forward, future work may hopefully apply this knowledge in developing a solution that can hopefully aid developer workflows by identifying and generating tests for semantic conflicts.

\section{Future Work}

During the course of our research we found many possible future avenues of research based both on our own work and existing research. Future work which explores these could improve
the quality of our prompting techniques and eventually developed a solution to automatically generate appropriate unit tests when a semantic merge conflict is detected in a commit.

Future improvements to prompting include:

- Improving understanding of semantic conflicts: As the LLM demonstrated an association between unintended software faults and conflicts, simple conflicts were ignored.
Possible work could focus on providing k-shot learning to improve this capability, or avoid the usage of ``conflict'' in favour of more direct or neutral terms, such as prompting
for the existence of lost or emergent behaviour.

- Prompt automation and picking necessary information: For our purposes, prompts were manually written. To fit within token limits and to avoid confusing the LLM, informed was cropped to
the necessary: only the class were the main affected method was present was given. In the case of real-world scenarios, diffs were also cropped, as they included a lot of extraneous information.
In an automated solution, investigation needs to be done on identifying the necessary elements required for the prompt and how to automatically extract them.

- Test generation with context: Throughout all our test generations, compilations tended to fail due to small errors, particularly with constructors, field access and method calls.
Providing context by prompting with an existing suite has shown to be a reliable solution and would also aid in following stylistically conventions a project may be using.
This could also be helpful with other issues, such as using mocking and more complex techniques when and only when it is necessary.

- Integration with Changes-Matcher\Citet{kn:nuno}: Changes-Matcher detects semantic conflicts by comparing merges to common patterns, such as Change Method, Parallel Field and others
seen throughout our work. Integrating these, possibly with a prompt to explain the specific type of conflict present could be helpful in conflict explanation.
Furthermore, Changes-Matcher outputs a DSL which provides information on how to test the conflict (indicating which methods should be called directly and indirectly). This could possibly
also improve test generation.

\subsection{Development Plan and Tool Functioning}
todo{
Development of the solution will go through several stages. Firstly, the process which has already been in progress, is the initial evaluation of LLM's and prompt techniques. After acquiring the list of subjects to test and deciding on the LLM, we can systematize this to develop the prototype solution. From here we start developing a tool that can automatically generate a prompt, get a test from the LLM and then run it. This prototype tool can then be further augmented, by applying corrections for tests that may fail to compile. \Cref{fig:tool} details the expected functioning of the tool once completed. A final period of evaluation and observations will compare our results to previous work and reflect on possible further improvements.
}
\begin{figure}
    \centering
    \includegraphics[width=1\linewidth]{figures/tool.pdf}
    \caption{Functioning pipeline of proposed tool.\todo{good enough? check svg better}}
    \label{fig:tool}
\end{figure}

%%----------------------------------------
%% Final materials
%%----------------------------------------

%% Bibliography
%% Comment the next command if BibTeX file not used
%% bibliography is in ``myrefs.bib''
\PrintBib{myrefs}

%% 2021-07-20: change
%% comment next 2 commands if numbered appendices are not used
%\appendix
%\chapter{Extra stuff} \label{ap1:Lorem}
\todo{better name for this chapter}

\section{Exploratory evaluation of ChatGPT's capabilities}

\subsection{Fabricated Examples}

In the initial step of work, we superficially explored ChatGPT's ability to generate tests for an example conflict of Point, where a distance method is altered from euclidean to manhattan in one branch and in the other branch, a move method is changed from using the value 1 for x and y movement to using the result of the distance calculation.

We tested two frameworks, first just asking for a test, with prompts based on the testing indications given by the DSL for the case:

\begin{itemize}
  \item A Dependency Based semantic conflict was possibly introduced in a 3-way merge. Develop a test for the class Point, that covers the methods move() and distance(), without calling distance() directly.
Before the merge, the class under test was: [base Point]
After the merge, it was: [merged Point].
  \item A Dependency Based semantic conflict was possibly introduced in a 3-way merge. Develop a test for the class Point, that covers the methods move() and distance(), without calling distance() directly.
Before the merge, the class under test was: [base Point]
In the branch A it was changed to: [A Point]
In the branch B it was changed to: [B Point]
After the merge, it was: [merged Point].

\end{itemize}

For this, the LLM simply took one version and created tests taking it as correct behaviour. In the first case, for Base and in the second for Merge. This is not ideal, as the first does not allow us to distinguish if the behaviour changed due to merging, or just do to changes in the branches. The latter takes merge as correct behaviour and will thus always fail.

Other tests involved first asking for an explanation if there was a merge conflict there, before asking for a test

\begin{itemize}
  \item We have done a merge on a piece of code.
Before the merge, the code was: [code]

In the branch A it was changed to: [code]

In the branch B it was changed to: [code]

After the merge it was: [code]

Do you believe there could be a merge conflict here? Where? Explain why.
  \item We have done a merge on a piece of code.
  
Before the merge, the code was: [code]

In the branch A it was changed to: [code]

In the branch B it was changed to: [code]

After the merge it was: [code]

Do you believe there could be a merge conflict here? Where? Explain why. Pay attention to both additions and modifications.
  \item We have done a merge on a piece of code code.
  
Before the merge, the code was: [code]

In the branch A it was changed to: [code]

In the branch B it was changed to: [code]

After the merge it was: [code]

Do you believe there could be a *semantic* merge conflict here? Where? Explain why. Pay attention to both additions and modifications.

\end{itemize}

The first two prompts failed. Indeed for prompt ChatGPT only mentioned the changes in move, but even after prompting it to pay attention to distance, it failed to identify the conflict. For both, it seemed to not understand what was being referred to when speaking of conflict, describing all changes as a "conflict". Thus it would say, there is a conflict between whether to use euclidian and manhattan distance and there is a conflict on how the move function works.

The last prompt given, highlighting the *semantic* aspect of the merge conflict, yielded good results, with the conflict being perfectly described, as seen in \Cref{fig:semconf}. After prompting for a test, it generates an appropriate one: it tests if the move function updates Point appropriately based on the euclidean distance. Thus it fails for base and branch B, where movement is still done with the value of 1, it passes for branch A as there move is based on distance; and it fails in merge, as now movement is being done with the distance calculation, but the calculation is now manhattan.

\begin{figure}
    \centering
    \includegraphics[width=0.75\linewidth]{figures/image.png}
    \caption{ChatGPT description of the semantic conflict.\todo{can we have a white-background of it? or having this in a \texttt{response} environment.}}
    \label{fig:semconf}
\end{figure}

In simple fabricated scenarios, where simple conflicts were added to existing software solutions, ChatGPT showed ability to identify and describe the semantic conflict. Despite this, test generation remained complicated and few of the successful identification of semantic conflicts yielded working tests. The final prompt follows.

\begin{quote}
Generate a Junit unit test to identify this semantic conflict, knowing what you do now. The test must compile without errors and require no further alterations. It should require no further dependencies and import all classes correctly.
\end{quote}

While adapted to avoid common pitfalls, the tests generate still suffered from basic issues such as missing imports, which could be mitigated by prompting the LLM for correction automatically. More complex issues of implementation were present, such as calling the base function instead of dependent, wrong usage of construction and function returns, parameter types or unnecessary mocks. Further, when prompting for correction, the LLM often explains that the developer should correct this, thus further work needs to be done to ensure the tool does all the work itself.

\subsection{Real-World Merge Conflicts}

Prompting in real-world examples, or more complex fabricated examples, has produced far worse results. Several refinements were made to the prompts, most significantly: the usage of "git diff" to highlight the specific changes in branch A and branch B, the explanation of the conflict present and the specification of the target method where the conflict is evident. These modifications were largely unsuccessful and did not lead to identification of any significant amount conflicts.

We also had to reckon with size limits for messages. Thus, when necessary, we started with an explanatory prompt and then fed the information step by step. However, it remained crucial to remind the LLM of the goal in the last message.

\begin{quote}
We have done a merge on a piece of software and introduced a semantic conflict of types: "Update two different dependencies of a method or update one method and concurrently update one of its dependencies" and "Concurrent changes to the same method". I will now show the base commit, the diff in branch A, the diff in branch B, and the final merge version in 4 separate messages. At the end I want you to explain why and where the semantic conflict is present.
\end{quote}

Common issues are confusion between textual and semantic merges, which can be mitigated by clear explanation of what a semantic merge is; hallucinations of features not present in the code or hallucinations of changes where changes were not made; lack of focus on the methods where conflict is evident, despite reiterations; loss of focus when prompt has to be split into several messages. 
Rare close successes occured, particularly in cases where the conflict was asserted to be there in the prompt and could simply be described by explaining the changes made, such as parallel changes to collections initialization, as seen in \Cref{fig:cantfind}.

\begin{figure}
    \centering
    \includegraphics[width=1\linewidth]{figures/almostsemantic.jpg}
    \caption{ChatGPT describes the parallel changes as the origin of the conflict, but falls short of describing the emergent behaviour (in this case as simple as the different value returned by size()).\todo{can we have a white-background of it? or having this in a \texttt{response} environment.}}
    \label{fig:cantfind}
\end{figure}

While we found that ChatGPT could not identify the semantic conflicts present in real software solutions, in some of \citet{kn:nuno}'s fabricated scenarios, these were correctly explained. Part of the difference between real-world examples and fabricated scenarios may come down to the information given. The fabricated scenarios were accompanied with a description of the specific semantic conflict present and the changes made perfectly reflected the description, with clear modifications and no extraneous changes. Thus it is possible collecting and offering that information with real-world scenarios may improve the ability of LLMs in this regard, but the higher "noise" of these scenarios may be too disruptive in this regard.

Another factor in consideration is the issue of dependencies, as so far testing had focused on just a unitary class. Given that semantic conflicts can involve interactions between classes and subclasses or other dependencies, it is relevant to provide further information. Initial tests just added one dependency, whether by calling a class methods or due to a inheritance relationship. An example, with the addition of an illustrative example of a semantic conflict follows.

\begin{quote}
We have done a merge on a piece of software and introduced a semantic conflict of type Parallel Changes in Method.

Semantic conflicts occur when concurrent and syntactic-correct changes in different regions of a source file or different files cause the software system to misbehave. For example, suppose there is a Java class `Point` with a method `distance()` that computes the Euclidean distance of a Point to the origin and Bob decides to modify `distance()` so it computes instead the Manhattan distance. At the same time, Alice, not aware of Bob's changes, creates a new method `move()` that uses `distance()` to calculate the Euclidean distance. Then, the changes of both developers are merged. As Bob and Alice did not modify the same lines of code, there is no textual conflict. There is neither a syntactic conflict as the merged code still compiles. However, the program now has an unexpected behaviour. The `move()` method introduced by Alice no longer moves a Point an Euclidean distance (as Alice was expecting) but rather moves a Manhattan distance

The affected declaration is copyWithDefaults().

A first message will detail the class before the merge, the diffs for both branches and the class after the merge. A second message will have a dependent class, whose methods indirectly call copyWithDefaults(). After I send these next two classes, identify and describe the semantic conflict.    
\end{quote}

A more complex evolution on this idea consisted of providing textual representations of UML graphs, such as plantUML call and structure graphs. These however proved to be complex in their own regard as they often induced the LLM to "forget" previous information and its goal, possibly due to their large size. A factor to consider, too, was that the information provided had to be limited in depth since, depending on the size of the software solution, a complete graph would be of extremely unwieldy size. Through some effort, we could get the LLM to recognize both the diagram and the class information, namely, by offering the diagram first.

\begin{quote}
We have done a merge on a piece of software. In this, we introduced a semantic conflict on the method dominates(State| State) of the class OpenTripPlanner. In a first message I will send a call diagram, in plantUML format. In a second message I will send the original class, the differences in the branches and the merged class.  At the end you must identify and explain the semantic conflict.
\end{quote}

However, it was particularly necessary to frequently refocus and concentrate the LLM. For example, after sending the diff and class information, reminding:

\begin{quote}
Explain why and where the semantic conflict is present, taking into account all the information provided in the last 2 messages. As mentioned before, focus on dominates(State| State). Make sure to mention information from the call diagram, if it is relevant to understanding the conflict.
\end{quote}

In the end, we were able to get the LLM to both recognize the merge information and the call diagram focused on the affected declaration and obtain a correct description of both the meaning of the diagram and the changes made in the merge. This however did not entail any improved description of the conflict, for all the cases tested.

Underscoring these experiments with real world test cases is an overlying issue: to properly identify a semantic conflict, we may have to analyze and take into account a large amount of data across several components of the software, while at the same time focusing on the specific components that were changed. While this is natural to a human, there is no clear way to select which information to feed and which to not in the LLM prompt: while a wider breadth of information may be necessary to identify the conflict, it is extremely likely to just induce further confusion and lead ChatGPT to focus on the wrong thing, or confuse different parts of code or even lose track of the task at hand as mentioned before.

\subsection{Test Generation without Conflict Understanding}

Given the difficulty of getting ChatGPT to provide accurate descriptions of the semantic conflicts, it was worth exploring whether the conflicts can be made evident just by generating tests targeting affected methods and classes. Several strategies were adopted: the prompting with the class after merge, the prompting with the class after merge and branch diffs and prompting with the class after merge and a structure graph. In most cases, where existing test suites where available, such information was also provided, with the prompt being reformulated to ask for its extension, rather than generation from scratch, as this shown to be important before~\cite{kn:githubcopilot}.

For 15 tests of 5 different projects, only 2 compiled, before and after prompting for corrections. One of these tested trivial additions to an array in both branches and the other identified a behaviour introduced in branch B that successfully carried over to merge, thus not a conflict. Positively, ChatGPT could understand the significance of this test working in branch B and merge but not A, as seen in \Cref{fig:testexplain}, suggesting it could understand the significance of a test failing for merge but not upper branches, which would indicate conflicts.

\begin{figure}
    \centering
    \includegraphics[width=1\linewidth]{figures/testexplain.jpg}
    \caption{ChatGPT explanation of meaning of failures and passes of test in different branches.\todo{can we have a white-background of it? or having this in a \texttt{response} environment.}}
    \label{fig:testexplain}
\end{figure}

Common issues affecting test generation were:

\begin{itemize}
  \item Faulty importing/setup: Particularly relevant in the OKHttp case, the tool was patently unable to correctly call imports, just omitting them, even when expressly being given the package names and being told to import it.
  \item Invalid access: The LLM was generally unable to distinguish between private and public methods and frequently made attempts to invoke or access private methods and fields.
  \item Method/Field/Class Hallucination: ChatGPT frequently invoked non-existent methods, fields or even classes. In some classes this could also be related to matters of access, as it generated getter calls for private fields.
  \item Incomplete/Template Tests: Despite being prompted explicitly to generate complete tests that should compile, in several cases tests were generated with incomplete template helper methods and classes.
\end{itemize}

While feeding compilation error outputs could be a solution for these, in practice no test that failed to compile was fixed in this manner, as changes made did not fix or ignored the problems present. In some cases, while the logic of correction was sound logically, they were not helpful in the context of automatic test generation. For example, a "MockConverter" class was used in a test for the Retrofit project. Upon prompting for a fix, an import was added, which still naturally was non-functional. Upon prompting for another fix, ChatGPT provided a basic empty template for the MockConverter class. While a correctly implemented MockConverter class would fix the issue and allow the test to compile, the solution here would have been to drop the usage of the MockConverter class in the first place

Experiments were made to employ the usage of vector indexing to boost the capabilities of understanding code. The theory was that, by indexing the entire repository of code, we could proceed with prompting without having to decide which specific blocks of code should be required information in the prompt and that, during generation, the LLM could correctly identify the chain of dependencies, and which methods and parameters should be employed to properly test our desired methods. In this process we employed llama\_index, but the results fell short: due to an observed worse ability to generating code, focus was placed on the explanation of what should be done, with the prompting: "We want to create an extensive testing suite for the method [method] of the class [class]. How should we approach this? Which calls should we make, which parameters should we try and which results should be expected?"
Results were generic, many times not even referring to aspects of the code itself and still prone issues of hallucination, specifically the making of references to non-existent methods.

\section{Comparison of Developed Prompts with State of the Art}

To evaluated our progressively developed prompt with existing state of the art, we selected 5 prompts from comparison, henceforth referred as 1 \cite{kn:chattester}, 2 \cite{kn:siddiq2023empirical}, 3 \cite{kn:gptunitbra}, and 4 \cite{kn:chatunitest}. Our prompt, in turn, was 2-step prompt as follows:

\begin{quote}
The following class was altered in a merge, specifically the METHOD method. Analyse it and what it does, with focus on the method and its usage: CLASS INFO
\end{quote}

\begin{quote}
You are a Java developer. Due to changes in the METHOD method, you've been asked to write a complete test suite to identify possible errors introduced due to the changes. Write a junit test suite for the method. All classes must be correctly imported. The tests must compile without errors. The tests must be complete and require no modification and addition. No explanation needed.
\end{quote}

As subjects, we selected the basic fabricated Point class example, Antlr4, whose testing simply requires the length of a returned list and OkHttp, a more complex example requiring mocks and reflection for testing.
For each prompt, we generated 3 times and selected the best result for comparison.

For Point, we made the following observations:

Our prompt generated a suite with 3 tests, one of which had an incorrect assertion (A Point with coordinates -3,-4 was expected to move to 4,5 when it should move to 4,3). Despite this one wrong test, the correct ones successfully identified the conflict, as they tested manhattan distance movement. For an earlier branch, with euclidian distance movement, they would fail, showing the conflict.

Prompt 1 called the distance() method to identify where the Point should be and set assertions accordingly. While this tested the move method, it could not identify the conflict, as the assertions were dependent on the behaviour of distance(), as it changed so did they.

Prompt's 2,3,4 all made the same mistake: Starting from a point 3,4, they expected it to move to 7,8. As distance is 7, it should actually move to 10,11. Notably, Prompt 3 and 4 generated tests for distance and correctly identified it as 7.

For Antlr4, our prompt ran into errors: it could not correctly initialize the class, as the CodeGenerator object called by the constructor was incorrectly created. It also severely undercounted the number of keywords, leading to an incorrect assertion.
Prompt 1 showed improvement over ours, as it understood a null could be used in place of CodeGenerator, as it would not be used for our purposes. Despite this, it failed in the logic of the test, calling badWords unnecessarily: this is a private method that is already called by the getBadWords method we are testing. This same error was present in Prompt 2.
Prompt 3 generated nonsensical tests, just checking if the result of methods called was false. While it tested non-existent methods, it did not tested getBadWords as we desired.
Prompt 4 also called addBadWords despite not needing it, but as it was instructed to use Mockito, it avoided the error of calling a private method by using reflection. The rest of the test logic and the assertion was correct.


OkHttp should be the hardest to test, as the method under test and the method which calls it are private. The value returned (which we seek to test) is never publicly available.
Thus all prompts try and fail to call the private method. The exception is Prompt 3 which, by it's nature of generating for all methods rather than being told to generate for a specific one, avoids private methods. Surprisingly, Prompt 4 which had shown ability to use reflection in the previous subject, failed to produce a satisfactory result. Also of note is Prompt 2, which calls for 10 tests to be generated: in this case, to reach this 'quota', each parameter of the copied client was tested in a entirely different test, rather than just in a different assertion.
Despite failing, due to previously mentioned issue with access, Prompt 1 notably produced the best assertions, as it tested both functionalities of the method: the correct copying of defined parameters, and the returning of defaults for parameters that were not set.


%% Index
%% Uncomment next command if index is required
%% don't forget to run ``makeindex thesis'' command
%\PrintIndex

\end{document}
