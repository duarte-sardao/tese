\chapter{Introduction} \label{chap:intro}

The purpose of this chapter is to introduce the motivation for the work, briefly describe the problem at hand and outline the work that will be developed, as well as the structure of the thesis.

\section{Motivation} \label{sec:motivation}

Collective software development requires the handling of merge conflicts, as conflicts between parallel work arise. Called merge conflicts, as they arise when this parallel work is merged, they vary in their difficulty of detection.
Common textual conflicts, where the same line is altered by multiple people, are automatically detected by version control systems, allowing amendments to be easily made. However not all conflicts are this easily detected and their introduction bring with it the addition of software bugs to the system. Semantic merge conflicts, in particular, are hard to detect, both by software and human review and remain a hard to solve situation.
The recent revolution in the field of Large Language Models (LLMs) may prove to add a valuable tool in tackling this.


\section{Problem} \label{sec:problem}

To manage the concurrent work of several developers in software projects, it is common to employ \emph{version control systems} (hereby referred as \emph{VCS}), which can be defined as "a system that manages the development of an evolving object. In other words, it is a system that records any changes made by the software developers." ~\citep{kn:vers_review}.

A significant task of version control is managing access to shared resources.
In the "classic scenario" a lock-modify-unlock paradigm was adopted, where a given file would be locked for modification while it is being modified, thus ensuring each resource can only be handled by one actor at a time. VCS's however, generally implement a copy-modify-merge mechanism: concurrent work can done on a resource, with joining the parallel work together handled by merges afterwards, with two "branches" of work merged into one. ~\citep{kn:vers_ott}

Merge conflicts arise when parallel work can't be clearly merged. Of this, several different types exist, as summarized by Tom Mens ~\citep{kn:tmens}:

Textual conflicts occur when the same textual elements of code are modified in both branches of a merge. For example, when the same line of code is modified by two people in their respective branches.

Syntactic conflicts arise from parallel changes that when merged don't generate textual conflicts, but the resulting merge creates code that is invalid given the languages rules. For example, programmer A renames a variable, while programmer B uses the variable somewhere (with the original name). There's no textual code, but the code won't compile due to the usage of an uninitialized variable.

Finally semantic conflicts occur when parallel changes don't trigger any conflict and are syntactically valid, but the resulting code doesn't behave as expected, or exhibits lost or new unexpected behaviour.

Most VCS's, such as Git, implement textual merge tools (ergo, they can only identify textual conflicts), however there are specialized tools that handle other types of merges: for example Turbomixer for syntactic merges.~\citep{kn:tmens} This focus on textual merging is generally fine as around 90\% of conflicts are textual ~\citep{kn:lcsd} and syntactic conflicts are easy to identify after merges, as errors will be clearly indicated and programs won't compile.
Semantic conflicts remain as both undetected by VCS's and hard to detect after merges. Thus, identifying methods to automatically identify and highlight semantic conflicts in merge commits has been a persistent problem and a source of study in this field. ~\citep{kn:nuno} ~\citep{kn:leuson} ~\citep{kn:leuson2}



\section{Goal and Approach} \label{sec:approach}

In previously developed work, a Domain-Specific Language was designed which describes the changes made in a three-way merge.~\citep{kn:nuno}. This DSL provides a summarized report of a three-way merge. We will feed the DSL reports into ChatGPT and induce it to generate unit tests based on them which should help assess the presence of a semantic conflict originating from the merge.






\section{Thesis Structure} \label{sec:struct}


Lorem ipsum