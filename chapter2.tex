\chapter{Related Work} \label{chap:sota}

\section{Introduction}

 The following chapters introduces the two main topics dealt with in this dissertation (merge conflicts and the usage of large language models for software verification and test generation). They explore related work and how we can build on it to develop our approach.

\section{Merge Conflicts and Semantic Conflicts}

The study of merge techniques and conflicts has a long history, likely even predating the specific terminology itself. Thus there is a large corpus of work to explore.

\subsection{Merge Conflicts}

\subsection{Detecting Semantic Conflicts}

Several related work exists, seeking to identify methods that can more systemically identify adulterated behaviour arising from semantic conflicts.

\subsubsection{Automated Behaviour Change Detection}

By Da Silva et al, we find an attempt at identifying cases of semantic conflict by applying automated behaviour change detection. ~\citep{kn:leuson} In summary, with a base commit B, a left L, right R and merge M, they observe that a generated unit test that passes in L but fails in B partially reveals the effect of the changes made in that branch. If the test then fails in M, it is likely the changes made in R interfere. To generate unit tests they used EvoSuite and Randoop. ~\citep{kn:randoop}
~\citep{kn:evosuite}

In their analysis "the test generation tools can detect interference in only four out of 15 changes on same declarations(in three merge scenarios) that in fact suffer from interference between the integrated changes.". While this is a very modest rate, it displayed no false positives and thus could likely be integrated in a testing process to prune possible merge conflicts early, or further studied and refined.



\subsubsection{Unit Tests}

In identifying the presence of semantic conflicts in a merge conflict, developed solutions have focused on the automatic generation of unit tests. Building upon their previous work, Da Silva et al ~\citep{kn:leuson2} worked with this approach: by proposing SAM (SemAntic Merge), a tool that generates tests upon merges in Java. In summary: "SAM first applies a textual merge tool to integrate the changes. In case no textual conflicts are reported, SAM builds the four program versions
associated with a merge scenario— a quadruple (Base, Left, Right, Merge)
formed by a merge commit (Merge), its parents (Left and Right), and a Base
commit— optionally applying source code transformations that might increase
program testability and feed the test generation tools with objects serialized
during the execution of existing project tests." After applying four test generations tools: Evosuite ~\citep{kn:evosuite}, Differential Evosuite, Randoop ~\citep{kn:randoop} and Randoop Clean, their own adapted version of Randoop, SAM "runs the generated tests against the four program builds, collects test failure information, interprets
that with our interference criteria heuristics, and finally reports detected
conflicts." ~\citep{kn:leuson2}.

With detections in 9 out of 28, it shows improvements over previous work: the authors specifically highlight the best performance when combining tests from only EvoSuite and Differential EvoSuite. In both cases, they highlight the ability of transformations (for example, making private fields public) to increase testability, showing moderate improvements in some tested scenarios: "We observe that the adoption of Testability Transformations help the tools
to detect behavior changes. In the same direction as for semantic conflict
detection, 20 additional changes are detected with testability executables,
when compared to the original executables (only 69 changes detected). From
the 19 false-negative cases observed in the experiment, we could detect
behavior changes in 11 of them. However, the reported behavior changes are
not caused by the changes involved in the semantic conflicts. So we can’t say
that the tools were close in these cases."  ~\citep{kn:leuson2}.


Nuno Castanho has proposed the tool UNSETTLE (aUtomatic uNit teSt gEneraTion for semanTic confLict dEtection) ~\citep{kn:nuno}. This tool is composed of two modules:

Changes-Matcher identifies the possible presence of semantic conflicts, by first computing the changes between different versions (base and variants) and then comparing it to a set of patterns describing common sources of conflicts as a base. From this it generates a DSL file, highlighting which methods and classes should be put under test to identify the conflict.

The second module is the test generator, a modified version of EvoSuite that takes the previously created artifact as an input to guide test generation.

Of particular interest to us is Changes-Matcher, as this is also the starting point for our work, with the usage of a LLM over EvoSuite for test generation instead.

\section{Test Generation}

\subsection{Traditional Automatic Test Generation}

~\citep{kn:randoop}
~\citep{kn:evosuite}
~\citep{kn:evosuiter}
~\citep{kn:pynguin}

\subsection{Test Generation with LLM's}

The recent explosion in complexity and popularity of LLM's has suscitated developer interest in their abilities with regards to accelerate and automate software engineering. Angela Fan et al identified that by 2023 3\% of pre-prints were related to Large Language Models and 11\% of those related to their use in software engineering.\cite{kn:angela} Particularly relevant is their ability to generate tests, with an expectation that they could achieve better coverage, correctness and readability than previous techniques of automated test generation.\cite{kn:junjiewang}
In comparison to traditional suites for automated test generation, such as EvoSuite, Palus, Randoop, and JTExpert, ChatGPT has shown, given right tuning of temperature settings, to show equivalent robustness.\cite{kn:gptunitbra}
The maximization of ChatGPT's abilities with regards to test generation has been explored: techniques such as prompting the LLM for a explanation of what the code is intending to do \cite{kn:nuances} and feeding error messages from codes that fail to compile or execute has intended back to the LLM for correction \cite{kn:chattester} have shown an amazing capacity for test generation, given the right prompting.


\section{Conclusions}


The related work aids our own work in several ways: the work explored as regards unit test generation can provide show us a baseline functioning of the idea we're trying to implement. Particularly, we can use their statistics regarding percentage of conflicts found to assess whether our own solution is an improvement with regards to previous work.

Crucially important too is the work of Nuno Castanho and, as previously mentioned, the developed Changes-Matcher. ~\citep{kn:nuno}.

The study of works on LLM test generation should also prove useful in the future, allowing us to get a better understanding of their functioning and capabilites, as well as providing ideas and prompt generation techniques which we may have to apply in the future.