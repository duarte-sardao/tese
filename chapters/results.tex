    \chapter{Experimental results}\label{chap:results}

This chapters answers \textbf{RQ1 – RQ?}, which study \todo{???}.

\section{RQ1}\label{sec:results:rq1}

\begin{table}[t]
\centering
% \tabcolsep=1cm
% \renewcommand{\arraystretch}{0.90}
\begin{tabular}{@{\extracolsep{\fill}} lrrrrr} \toprule
                 & A & B & C & D & E \\
\midrule
Point            & 3.0 & 1.0 & 0.0 & 0.0 & 0.0 \\
Fabricated (RO)  & 3.0 & 1.0 & 1.0 & 3.0 & 1.0 \\
Fabricated (OAC) & 3.0 & 1.0 & 1.0 & 3.0 & 1.0 \\
Cart (CM)        & 3.0 & 1.0 & 1.0 & 2.0 & 0.0 \\
Cart (PF)        & 1.0 & 1.0 & 2.0 & 3.0 & 1.0 \\
Cart (PM)        & 3.0 & 1.0 & 1.0 & 3.0 & 3.0 \\
Antlr            & 1.0 & 1.0 & 0.0 & 0.0 & 0.0 \\
OkHttp           & 3.0 & 1.0 & 1.0 & 0.0 & 0.0 \\
\midrule
\textit{Median}  & 3.0 & 1.0 & 1.0 & 2.5 & 0.5 \\
\bottomrule
\end{tabular}
\caption{Median values per subject and metric.\label{tab:results:rq1}}
\end{table}

\Cref{tab:results:rq1} summarizes the metrics achieved per subject.
The complete prompts, with results and evaluated metrics, can be found in the attachments \todo{Which one? Do not forget to cref it here}.

A first observation to make is the variability of results, which highlights the importance of running 3 trials for our evaluation.
For example, the point class had high variation: two of the results failed to identify any conflict at all, but one of the three not only identified, but quite accurately describe it's origin and results.

In the more complex fabricated examples, the biggest issue was the identification of the effects of the semantic conflict, namely that when an overriden function was removed or had its' access changed, the other function, with different behaviour, would be used instead. However, given the constraints of the prompt, which is limited to one class, the necessary information could not be transmitted and the LLM argued there would be compilation errors. 
Better results were observed when doing more manual, ad-hoc experiments, as we could identify exactly all the classes involved in the conflict and offer all the required information.

As we proceeded testing the "Cart" scenario, it was notable that in the cases where the conflict is identified, with varying exactness, it falls short of being able to explain the exact result of the conflict, namely that if an admin tries to checkout, there will always be a exception thrown. Instead, we see more vague declarations, such as "As a result, it may lead to unexpected behavior or exceptions during execution, especially if the admin status of a user is incorrectly handled" and unnecessary usage of expressions such as 'potentially' in "If `user.admin` is true, `total\_cost` will return 0, potentially causing the `checkout` method in version B to throw a `RuntimeException` even if `user.admin` is true".
This was a recurring trend that was observed throughout all tests, as can be observed by a low final average score for metric E.

In a scenario where detection failed we see an example of the LLM providing a grossly wrong explanation: "If the user is an admin and a discount code that results in a cost of 0 is applied, the exception will never be thrown because the `checkout` method won't be called.". As regards the conflict itself, this was only completely wrong assertion of it's kind: in general where answers failed they were either vague or omitted, not present. The only other major example of a factually incorrect answer came with OkHttp:
\begin{lstlisting}
### Explanation:
The conflict arises because both changes affect the same field (`hostnameVerifier`), but they prescribe different default values.
    
In the merge commit, the decision was made to adopt the change from Version B, which explicitly sets the default `hostnameVerifier` to `OkHostnameVerifier()`. This decision potentially overrides any behavior or configuration related to the default `hostnameVerifier` set in Version A.
    
If the intended behavior was to preserve the default `hostnameVerifier` as defined in Version A, then this would be considered a semantic conflict, as the behavior of the merged version may differ from the behavior of either Version A or Version B individually.
\end{lstlisting}
In this case, the changes made in only one branch are described as a conflict.

Another case of wrong information given by ChatGPT regarded the changes made, both in Cart(PF):
\begin{lstlisting}
- Version A introduces a new discount code called "megasaver" with a discount value of 0.4.
- Version B modifies the discount value of the "superpromo" code to 0.25 instead of 0.3.
------------------------------------------------------------------------
1. **Version A**:
   - It adds a new discount code `"megasaver"` with a discount of 0.4d.

2. **Version B**:
   - It changes the discount value for the `"superpromo"` code from 0.25d to 0.25d.
------------------------------------------------------------------------
- Version A adds a new discount code "megasaver" with a discount of 0.4.
- Version B modifies the discount of the "superpromo" code to 0.25.
\end{lstlisting}
As well as Antlr:
\begin{lstlisting}
1. **Diff A vs Base:**
    - In version A, the `reduce` keyword was replaced with `return` in the `python2Keywords` array.
 
2. **Diff B vs Base:**
    - In version B, the `del` keyword was added to the `python2Keywords` array.
-------------------------------------------------------------------------------
1. **Diff A vs. Base**:
    - Addition of `"return"` keyword in the `python2Keywords` array.
    - No conflicting changes with other methods or properties.
 
2. **Diff B vs. Base**:
    - Replacement of `"delattr"` with `"del"` in the `python2Keywords` array.
    - No conflicting changes with other methods or properties.
\end{lstlisting}
However this did not worsen it's ability to identify the conflict in the case of Cart, where the description of the conflict as in line with the changes made.
We observed both of these are "Parallel Field" semantic conflicts so it is possible the fact that the changes are being made to fields may have some relation to its' inability to express them correctly.

As the most complex example, OkHttp generated the least conclusive answers. In particular it tended to offer advice on how identify and diagnose semantic conflicts and make good merges. This is a behaviour that was also observed with other complex, real-world conflicts, during the period of ad-hoc testing.
\begin{lstlisting}
### Potential Semantic Conflict:
While there's no direct semantic conflict, there could be conflicts in the usage and interpretation of the `transports` field and its associated methods. For instance:
- How are these transports selected and utilized within the class?
- Do they introduce any performance or compatibility issues?
- How do these changes impact existing functionality and client behavior?
---------------------------------------------------------------------------------------
### Semantic Conflict Assessment:
Looking at the changes introduced in versions A and B, there doesn't seem to be a direct semantic conflict between them. Version A primarily deals with transport configuration and introduces a new field `transports` along with related methods. On the other hand, version B modifies the hostname verifier behavior.

However, when merging both changes, we need to ensure that the introduced hostname verifier (`OkHostnameVerifier`) is consistent with the overall configuration of the `OkHttpClient`. Specifically, we need to verify whether `OkHostnameVerifier` is compatible with the behavior and expectations of the existing codebase.
\end{lstlisting}


\subsection{Averages and Conclusion}

\todo{Summarize the following and move it to the beginning of \Cref{sec:results:rq1}.}

By averaging out the results for each metric, we can find where the systems excel and where they fail the most. Thus, we find:

- Changes between branches correctly identified: 2.5/3
ChatGPT is generally able to identify changes made between branches. In cases where they were incorrectly describe, the error were mostly in one branch, relating to altered fields. Despite this, we observed it can still describe conflicts, even when it fails to properly describe changes.

- No conflict misunderstanding: 1/1

In none of our examples did the tool mistake the kind of conflict under discussion. This might be because the prompt did not assert the existence of a conflict, allowing negative responses, whereas other prompts, which asked to find a conflict that did exist, lead the model to find any explanation, which often consisted of describing a textual conflict.

- Positive response: 0.875/2

The average here is slightly under 1, reflecting our observations: answers given tend to identify a conflict, but with uncertainty (with language such as "possible", "might"). In other cases, the cases where no conflict was found outweighed the cases where one was found with uncertainty.

- Origin of conflict described: 1.5/3

Identification of the origins of conflict falls squarely in the middle of our given range.  When we exclude 0 value results, the average is 2.77 instead. This indicates to us in nearly half the cases, the model failed to produce any description, but when it did, it was generally very accurate.

- Effect of conflict described: 0.667/3

With the relatively lowest value, we find that ChatGPT struggles with accurately describing the results of semantic conflicts, ie, how the behaviour is specifically altered and how the outputs change. Indeed, even when the origin of the conflict is found, an accurate description of the results does not necessarily follow. Excluding 0 value results, we find an average 1.6: unlike identification of the origin, which was generally reliable when present, identification of the effect was still often vague or inaccurate.


Overall, we can say ChatGPT can indeed identify and explain whether there is a semantic conflict in a merge. This comes with many caveats however, firstly being that identification does not necessarily lead to a proper explanation. This is not a particularly pressing issue, as a tool that identifies conflicts and then alerts humans, who can themselves describe and fix the issue would already be a great boon for efficient software development.

The bigger caveat is the situations in which it identifies the conflicts: simple fabricated scenarios, whose changes are simple to identify and immediately clearly related. In real world scenarios, where its not immediately obvious what functions the changes serve and how they relate to each other. To resolve this, it might necessary to work out how to efficiently feed the LLM so it can gain an accurate understanding of the whole system without losing focus on the specific task at hand.
